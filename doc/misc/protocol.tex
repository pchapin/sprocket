\section{Security Protocol}
\label{section-protocol}

We evaluate our security protocol in the context of a Dolev-Yao attacker \cite{something}. Such
an attacker has complete power over the network but no ability to defeat any cryptography nor
tamper with any node. \pnote{This isn't very realistic for a WSN since the nodes are physically
  vulnerable. However, it is a fairly standard model and probably reasonable for our purposes.}

The protocol consists of three parts
\begin{enumerate}
\item Distribution of certificates
\item Session key negotiation
\item Normal RPC messages
\end{enumerate}

The primary security property we wish to guarantee is: \textit{no service should be accessible
  by a client that is not authorized for that service}.

\subsection{Certificate Distribution}
\lable{section-certificate-distribution}

The certificate distribution protocol is secure against a Dolev-Yao attacker.
\begin{enumerate}
\item Certificates contain only public information. An adversary is free to read them.
\item Blocking certificates is safe due to monotonicity of $RT_0$. Modified certificates will
  fail to verify.
\item Certificates can be replayed with no ill effect (except possibly denial of service).
\item An adversary can send unrelated certificates although a denial of service attack is
  possible by flooding a receiver with useless certificates and pushing useful ones out of its
  certificate storage area.
\end{enumerate}

\subsection{Session Key Negotiation}
\label{section-session-key-negotiation}

Because of the high computational cost of public key cryptography, especially relative to the
abilities of a typical wireless sensor node, our system negotiates session keys for each dynamic
wire connection. Let $A_n$ denote a client node $n$ in security domain $A$, and $B_m$ denote a
server node $m$ in security domain $B$. For any particular client node a session key is
negotiated for each $(m, C, I, D)$ tuple. Since the session key only depends on the public and
private keys of the interacting security domains, the same session key may, in general, be
negotiated multiple times for different services or different neighbor nodes. However a client
can not use a previously negotiated session key to access a unauthorized service on the security
domain as an authorized service. This is because the server only records session keys for
authorized services; the lack of a recorded key for a particular service implies that
authorization has not yet been done.

The specific protocol is as follows
\begin{verbatim}

1. A_n -> B_m : A, (m, C, I, D)
2. B_m -> A_n : B, (m, C, I, D)
  A_n computes session key for SELF -> (m, C, I, D) communications

\end{verbatim}

In step 1 the client sends a message to the server indentifying itself along with the desired
service endpoint (as denoted by the $(C, I, D)$ triple. The server checks authorization of the
client key $A$ and then computes a session key using the Diffie-Hellman key agreement protocol.
The session key is recorded in a cache on the server node for future use.

In step 2 the server sends a message to the client indentifying itself and echoing the full
service endpoint. The client computes the session key and records it in its cache.

Notice that a man-in-the-middle attack is not possible here because the $RT$ trust management
system uses the public keys themselves as principals. If a middle node $M$ is able to negotiate
a session key with the server, then $M$ must be an authorized key and thus a legitimate user of
the service anyway.