\section{Related Work}
\label{section-related-work}

Extending wireless sensor network software platforms with support for
secure interactions between domains has been studied in previous
research on SSL for WSNs \cite{10.1109/WAINA.2009.47}. However, this
work was focused on extending the Internet to WSNs (a.k.a. ``IP for
WSNs''), whereas SpartanRPC is a more general system for enhancing
secure communications \emph{within} a WSN.  Research on WSN security
has also addressed secure routing \cite{senroute-ahnj03}, link layer
security \cite{karlog-tinysec-2004}, cryptography \cite{bertoni-2006},
key distribution \cite{camtepe-bulent-05}, and hardware issues
\cite{perrig-2004}. In contrast to these low-level systems, SpartanRPC
provides language-level abstractions for secure RPC services.  Perhaps
even more closely related in this same vein is a system for
establishing fine-grained, ``node-level'' policies in WSNs
\cite{Claycomb:2011:NNL:1889383.1889450}. However, this work is more
focused on group-based key negotiation and distribution, and while it
does offer a policy language, it is rooted in implementation details
and not a separable specification as in our system. Also, they do not
provide a language API for integrating their system into secure
applications as in SpartanRPC.

Previous related work also illustrates interest in and useful
applications of RPC in embedded networks. For example, the Marionette
system uses network layer RPC for remote (PC-based) analysis and
debugging of WSNs \cite{whitehouse-marionette-2006}. The Fleck operating
system provides a small pre-defined set of RPC services for WSN
applications, while the trustedFleck system extends this with a form a
secure RPC \cite{hu-secfleck-2009,Hu:2010:TTW:1806895.1806900}. S-RPC
provides an RPC facility for sensor networks that allows remote services
to be added to the system dynamically \cite{5766863}. SpartanRPC differs
from these systems in that it extends the nesC programming language
(unlike trustedFleck) to allow programmer definition of secure RPC
services (unlike S-RPC) that can be accessed by nodes within the network
itself (unlike Marionette). Our system is similar to and inspired by
TinyRPC \cite{may-tinyrpc-2007}, except the latter does not provide
security and has a different semantics that are not as expressive as our
approach.


Teeny\textsc{lime} allows application programs to access an abstract
``tuple space'' that is the union of tuple spaces on the local node and
the immediately neighboring nodes \cite{Costa:2007:PWS:1516124.1516153}.
This provides an alternative to RPC for uniformly accessing remote and
local data. However, interaction with the middleware is by way of a
dedicated API; there is no attempt to provide a true RPC mechanism. Also
Teeny\textsc{lime} does not address issues of access control.

Secure Middleware for Embedded Peer to Peer systems (SMEPP) is a general
framework for creating security sensitive applications from a
distributed network of embedded peers
\cite{Brogi:2008:SME:1363370.1363548}. SMEPP Light
\cite{Vairo:2008:SMW:1594978.1595054} is a reduced version of SMEPP to
address the resource constraints of wireless sensor networks. SMEPP
Light provides a publish/subscribe communication model using directed
diffusion to distribute ``events'' to all subscribers and symmetric key
cryptography to provide confidentiality and data integrity within a
group of nodes. However, SMEPP Light is not integrated into a
programming language and does not provide a remote procedure call
mechanism. Furthermore SMEPP Light only supports a simple model of
access control based on group membership.

High level macro\-programming languages such as Kairos
\cite{springerlink:10.1007/1150259312}, Regiment
\cite{Newton:2007:RMS:1236360.1236422}, and even Flask
\cite{Mainland-Flask-2008} provide a way to program the entire network
as a single entity. These systems attempt to hide not only the
inter-node communication from the programmer, but also the entire node
level programs. SpartanRPC operates at a much lower level and also,
unlike these macro\-programming systems, addresses access control issues
in networks containing multiple security domains.

Whole network programming of wireless sensor networks has also been
investigated using mobile agents in systems such as Agilla
\cite{Fok:2009:AMA:1552297.1552299} and Wiseman
\cite{Gonzalez-Valenzuela:2010:PMW:1891545.1891566}. However, like the
macro\-programming systems mentioned previously neither of these systems
address issues related to access control in the presence of multiple
security domains.

\section{Conclusion}
\label{section-conclusion}

We have designed and implemented SpartanRPC, a dialect of nesC with a
light weight, link-layer, secure RPC API. SpartanRPC is a middleware
technology supporting secure WSN applications comprising multiple
security domains. It is ideal for settings in which multiple networks
administered by distinct social entities cooperate to obtain a
holistic behavior. As discussed in \autoref{section-related-work},
currently no other WSN security architectures support multiple
security domains or principled techniques for communication between
them. Thus, SpartanRPC provides crucial tools for next-generation WSN
applications wherein multiple, distinct security domains interact.

RPC communication in our system is implemented using a modification of
existing nesC abstractions, specifically module wirings. In
SpartanRPC, module wirings connect to remote services dynamically.
Furthermore, these connections are mediated by authorization levels
specified by an access control policy defined in the trust management
language RT \cite{Li:DRBTMF}, and authorization is proved by
presentation of RT credentials by requesters. Our implementation is
based on public keys, supporting an open-world security model where
shared secrets need not be known a priori. Underlying security
protocols defend against man-in-the middle attacks through the use of
a Diffie-Hellman protocol, ensuring that only authorized principals
may access resources. We have reported on testing and performance
evaluations, providing evidence of the practicality of SpartanRPC in
its intended application space.  We have also used SpartanRPC to
implement a secure real-world WSN application for environmental data
collection, demonstrating the effectiveness of the API and of the
authorization logic.
