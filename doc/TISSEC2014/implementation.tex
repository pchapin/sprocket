\section{The SpartanRPC Implementation}
\label{section-implementation}

In this section we describe the implementation of the SpartanRPC system
using $RT_0$ trust management for authorization. We call our
implementation \Sprocket \cite{sprocket}. \Sprocket\ rewrites a
SpartanRPC program into a pure nesC program and provides a supporting
runtime system. Program rewriting converts remote duty postings into a
nesC messaging protocol. The main task of the runtime system is to
implement the encapsulated, underlying security protocols for
authorization of remote duty postings.

\subsection{Authorization and Security Protocols}
\label{section-security-protocols}
\label{section-underlying-protocols}

\Sprocket\ implements SpartanRPC authorization using a combination of
public and symmetric key cryptography. We use the TinyECC library
\cite{Liu-Peng-TinyECC-2008} for public key functionality, and AES
encryption for symmetric key functionality. TinyECC uses elliptic curve
cryptography for more efficient public key operations in WSNs. Using AES
has the benefit of hardware support on many current embedded platforms,
e.g.~those employing the Chipcon CC2420 radio.

There are three security protocols for authorized duty postings,
illustrated in \autoref{figure-sprocketrt-protocol}, that operate
asynchronously. First, a credential exchange protocol in which \RT\
credentials are communicated between nodes and authorization levels for
various entities are computed, i.e.~the \emph{minimum model} as
described in \autoref{section-security-extensions}. Second, a session
key protocol in which symmetric keys for multiple authorized service
invocations are computed between a client and server. And third, an
authorized service invocation protocol in which duty postings are
generated and checked. This decomposition of authorized service
invocation into three protocols supports efficiency especially through
the use of symmetric keys for multiple service invocations. Its
asynchronous nature is also appropriate in an asynchronous TinyOS
setting.

\subsubsection{Credential Exchange}

\begin{figure}[t]
  
\expandafter\ifx\csname graph\endcsname\relax
   \csname newbox\expandafter\endcsname\csname graph\endcsname
\fi
\ifx\graphtemp\undefined
  \csname newdimen\endcsname\graphtemp
\fi
\expandafter\setbox\csname graph\endcsname
 =\vtop{\vskip 0pt\hbox{%
    \special{pn 8}%
    \special{pa 200 700}%
    \special{pa 1000 700}%
    \special{pa 1000 200}%
    \special{pa 200 200}%
    \special{pa 200 700}%
    \special{fp}%
    \graphtemp=\baselineskip
    \multiply\graphtemp by -1
    \divide\graphtemp by 2
    \advance\graphtemp by .5ex
    \advance\graphtemp by 0.450in
    \rlap{\kern 0.600in\lower\graphtemp\hbox to 0pt{\hss {\small Certificate}\hss}}%
    \graphtemp=\baselineskip
    \multiply\graphtemp by 1
    \divide\graphtemp by 2
    \advance\graphtemp by .5ex
    \advance\graphtemp by 0.450in
    \rlap{\kern 0.600in\lower\graphtemp\hbox to 0pt{\hss {\small Sender}\hss}}%
    \special{pa 2400 700}%
    \special{pa 3200 700}%
    \special{pa 3200 200}%
    \special{pa 2400 200}%
    \special{pa 2400 700}%
    \special{fp}%
    \graphtemp=\baselineskip
    \multiply\graphtemp by -1
    \divide\graphtemp by 2
    \advance\graphtemp by .5ex
    \advance\graphtemp by 0.450in
    \rlap{\kern 2.800in\lower\graphtemp\hbox to 0pt{\hss {\small Certificate}\hss}}%
    \graphtemp=\baselineskip
    \multiply\graphtemp by 1
    \divide\graphtemp by 2
    \advance\graphtemp by .5ex
    \advance\graphtemp by 0.450in
    \rlap{\kern 2.800in\lower\graphtemp\hbox to 0pt{\hss {\small Receiver}\hss}}%
    \special{pa 200 1400}%
    \special{pa 1000 1400}%
    \special{pa 1000 900}%
    \special{pa 200 900}%
    \special{pa 200 1400}%
    \special{fp}%
    \graphtemp=\baselineskip
    \multiply\graphtemp by -1
    \divide\graphtemp by 2
    \advance\graphtemp by .5ex
    \advance\graphtemp by 1.150in
    \rlap{\kern 0.600in\lower\graphtemp\hbox to 0pt{\hss {\small Session Key}\hss}}%
    \graphtemp=\baselineskip
    \multiply\graphtemp by 1
    \divide\graphtemp by 2
    \advance\graphtemp by .5ex
    \advance\graphtemp by 1.150in
    \rlap{\kern 0.600in\lower\graphtemp\hbox to 0pt{\hss {\small Sender}\hss}}%
    \special{pa 2400 1400}%
    \special{pa 3200 1400}%
    \special{pa 3200 900}%
    \special{pa 2400 900}%
    \special{pa 2400 1400}%
    \special{fp}%
    \graphtemp=\baselineskip
    \multiply\graphtemp by -1
    \divide\graphtemp by 2
    \advance\graphtemp by .5ex
    \advance\graphtemp by 1.150in
    \rlap{\kern 2.800in\lower\graphtemp\hbox to 0pt{\hss {\small Session Key}\hss}}%
    \graphtemp=\baselineskip
    \multiply\graphtemp by 1
    \divide\graphtemp by 2
    \advance\graphtemp by .5ex
    \advance\graphtemp by 1.150in
    \rlap{\kern 2.800in\lower\graphtemp\hbox to 0pt{\hss {\small Receiver}\hss}}%
    \special{pa 200 2100}%
    \special{pa 1000 2100}%
    \special{pa 1000 1600}%
    \special{pa 200 1600}%
    \special{pa 200 2100}%
    \special{fp}%
    \graphtemp=.5ex
    \advance\graphtemp by 1.850in
    \rlap{\kern 0.600in\lower\graphtemp\hbox to 0pt{\hss {\small Client}\hss}}%
    \special{pa 2400 2100}%
    \special{pa 3200 2100}%
    \special{pa 3200 1600}%
    \special{pa 2400 1600}%
    \special{pa 2400 2100}%
    \special{fp}%
    \graphtemp=.5ex
    \advance\graphtemp by 1.850in
    \rlap{\kern 2.800in\lower\graphtemp\hbox to 0pt{\hss {\small Server}\hss}}%
    \graphtemp=.5ex
    \advance\graphtemp by 2.450in
    \rlap{\kern 0.600in\lower\graphtemp\hbox to 0pt{\hss {\small Node A}\hss}}%
    \graphtemp=.5ex
    \advance\graphtemp by 2.450in
    \rlap{\kern 2.800in\lower\graphtemp\hbox to 0pt{\hss {\small Node B}\hss}}%
    \special{pa 0 2300}%
    \special{pa 1200 2300}%
    \special{pa 1200 0}%
    \special{pa 0 0}%
    \special{pa 0 2300}%
    \special{da 0.050}%
    \special{pa 2200 2300}%
    \special{pa 3400 2300}%
    \special{pa 3400 0}%
    \special{pa 2200 0}%
    \special{pa 2200 2300}%
    \special{da 0.050}%
    \special{sh 1.000}%
    \special{pn 1}%
    \special{pa 2300 425}%
    \special{pa 2400 450}%
    \special{pa 2300 475}%
    \special{pa 2300 425}%
    \special{fp}%
    \special{pn 8}%
    \special{pa 1000 450}%
    \special{pa 2300 450}%
    \special{fp}%
    \graphtemp=\baselineskip
    \multiply\graphtemp by -1
    \divide\graphtemp by 2
    \advance\graphtemp by .5ex
    \advance\graphtemp by 0.450in
    \rlap{\kern 1.700in\lower\graphtemp\hbox to 0pt{\hss {\small Certificates}\hss}}%
    \special{sh 1.000}%
    \special{pn 1}%
    \special{pa 2300 1000}%
    \special{pa 2400 1025}%
    \special{pa 2300 1050}%
    \special{pa 2300 1000}%
    \special{fp}%
    \special{pn 8}%
    \special{pa 1000 1025}%
    \special{pa 2300 1025}%
    \special{fp}%
    \graphtemp=\baselineskip
    \multiply\graphtemp by -1
    \divide\graphtemp by 2
    \advance\graphtemp by .5ex
    \advance\graphtemp by 1.025in
    \rlap{\kern 1.700in\lower\graphtemp\hbox to 0pt{\hss {\small $(K_{cp}, C, I)$}\hss}}%
    \special{sh 1.000}%
    \special{pn 1}%
    \special{pa 1100 1300}%
    \special{pa 1000 1275}%
    \special{pa 1100 1250}%
    \special{pa 1100 1300}%
    \special{fp}%
    \special{pn 8}%
    \special{pa 2400 1275}%
    \special{pa 1100 1275}%
    \special{fp}%
    \graphtemp=.5ex
    \advance\graphtemp by 1.375in
    \rlap{\kern 1.700in\lower\graphtemp\hbox to 0pt{\hss {\small $(K_{sp}, C, I)$}\hss}}%
    \special{sh 1.000}%
    \special{pn 1}%
    \special{pa 2300 1825}%
    \special{pa 2400 1850}%
    \special{pa 2300 1875}%
    \special{pa 2300 1825}%
    \special{fp}%
    \special{pn 8}%
    \special{pa 1000 1850}%
    \special{pa 2300 1850}%
    \special{fp}%
    \graphtemp=\baselineskip
    \multiply\graphtemp by -1
    \divide\graphtemp by 2
    \advance\graphtemp by .5ex
    \advance\graphtemp by 1.850in
    \rlap{\kern 1.700in\lower\graphtemp\hbox to 0pt{\hss {\small Post + MAC}\hss}}%
    \hbox{\vrule depth2.450in width0pt height 0pt}%
    \kern 3.400in
  }%
}%

  \centerline{\raise 1em\box\graph}
  \vspace{2mm}
  \caption{SpartanRPC Security Protocol Elements}
  \label{figure-sprocketrt-protocol}
\end{figure}

\label{section-certificate-format}

SpartanRPC credentials are implemented as signed certificates. All
SpartanRPC-enabled nodes contain a certificate sender component and a
certificate receiver component to transfer certificates between nodes
and to verify them and interpret the credentials they represent. Both
components run as background daemons, performing their function
asynchronously. A SpartanRPC-enabled node is deployed with a collection
of certificates in read-only storage representing that node's
credentials, which are determined by some external means. Once the node
is booted, the certificate sender starts a periodic timer. When the
timer fires, the node link-layer broadcasts (i.e.~only to neighbors) all
certificates in its certificate storage that are mentioned in the
\texttt{enable} clauses of the dynamic wires used during program
execution. To prevent adjacent node certificate broadcasts from
colliding, the certificate broadcast interval is modulated randomly by
$\pm 10$\%. For example if the nominal broadcast interval is one minute,
the actual time varies randomly between 54 and 66 seconds.

The certificate distribution strategy is robust in the face of new nodes
being added to the network or intermittent radio connectivity. If a node
fails to receive certain certificates from its neighbors it will have
another opportunity to do so when those neighbors rebroadcast their
certificates. There is a trade off between the broadcast interval,
responsiveness, and network energy consumption. A short broadcast
interval allows authorizations to succeed ``quickly'' since neighbors
become aware of the necessary credentials early, but at the cost of
increased radio traffic and power consumption.

Once a newly received certificate has been verified, the credential it
represents is extracted and stored. The credential storage also contains
the $RT_0$ minimum model implied by the currently known collection of
credentials. Each time a new credential is added to storage, the minimum
model is updated. This is done by repeatedly applying authorization
logic inference rules implied by the credentials to the current model
until a fixed point is reached, i.e.~a logical closure \cite{Li:DCFTML}.
Thus, each node maintains a local view of authorization levels for
network entities based on received credentials.

\begin{figure}[t]
  \expandafter\ifx\csname graph\endcsname\relax
   \csname newbox\expandafter\endcsname\csname graph\endcsname
\fi
\ifx\graphtemp\undefined
  \csname newdimen\endcsname\graphtemp
\fi
\expandafter\setbox\csname graph\endcsname
 =\vtop{\vskip 0pt\hbox{%
    \graphtemp=.5ex
    \advance\graphtemp by 0.125in
    \rlap{\kern 0.000in\lower\graphtemp\hbox to 0pt{$A.r \leftarrow B.s \cap C.t$\hss}}%
    \special{pn 8}%
    \special{pa 0 500}%
    \special{pa 250 500}%
    \special{pa 250 250}%
    \special{pa 0 250}%
    \special{pa 0 500}%
    \special{fp}%
    \graphtemp=.5ex
    \advance\graphtemp by 0.375in
    \rlap{\kern 0.125in\lower\graphtemp\hbox to 0pt{\hss 4\hss}}%
    \special{pa 250 500}%
    \special{pa 1000 500}%
    \special{pa 1000 250}%
    \special{pa 250 250}%
    \special{pa 250 500}%
    \special{fp}%
    \graphtemp=.5ex
    \advance\graphtemp by 0.375in
    \rlap{\kern 0.625in\lower\graphtemp\hbox to 0pt{\hss $A$ (40)\hss}}%
    \special{pa 1000 500}%
    \special{pa 1250 500}%
    \special{pa 1250 250}%
    \special{pa 1000 250}%
    \special{pa 1000 500}%
    \special{fp}%
    \graphtemp=.5ex
    \advance\graphtemp by 0.375in
    \rlap{\kern 1.125in\lower\graphtemp\hbox to 0pt{\hss $r$\hss}}%
    \special{pa 1250 500}%
    \special{pa 2000 500}%
    \special{pa 2000 250}%
    \special{pa 1250 250}%
    \special{pa 1250 500}%
    \special{fp}%
    \graphtemp=.5ex
    \advance\graphtemp by 0.375in
    \rlap{\kern 1.625in\lower\graphtemp\hbox to 0pt{\hss $B$ (40)\hss}}%
    \special{pa 2000 500}%
    \special{pa 2250 500}%
    \special{pa 2250 250}%
    \special{pa 2000 250}%
    \special{pa 2000 500}%
    \special{fp}%
    \graphtemp=.5ex
    \advance\graphtemp by 0.375in
    \rlap{\kern 2.125in\lower\graphtemp\hbox to 0pt{\hss $s$\hss}}%
    \special{pa 2250 500}%
    \special{pa 3000 500}%
    \special{pa 3000 250}%
    \special{pa 2250 250}%
    \special{pa 2250 500}%
    \special{fp}%
    \graphtemp=.5ex
    \advance\graphtemp by 0.375in
    \rlap{\kern 2.625in\lower\graphtemp\hbox to 0pt{\hss $C$ (40)\hss}}%
    \special{pa 3000 500}%
    \special{pa 3250 500}%
    \special{pa 3250 250}%
    \special{pa 3000 250}%
    \special{pa 3000 500}%
    \special{fp}%
    \graphtemp=.5ex
    \advance\graphtemp by 0.375in
    \rlap{\kern 3.125in\lower\graphtemp\hbox to 0pt{\hss $t$\hss}}%
    \special{pa 3250 500}%
    \special{pa 4000 500}%
    \special{pa 4000 250}%
    \special{pa 3250 250}%
    \special{pa 3250 500}%
    \special{fp}%
    \graphtemp=.5ex
    \advance\graphtemp by 0.375in
    \rlap{\kern 3.625in\lower\graphtemp\hbox to 0pt{\hss sig (42)\hss}}%
    \special{pa 4000 500}%
    \special{pa 4750 500}%
    \special{pa 4750 250}%
    \special{pa 4000 250}%
    \special{pa 4000 500}%
    \special{fp}%
    \graphtemp=.5ex
    \advance\graphtemp by 0.375in
    \rlap{\kern 4.375in\lower\graphtemp\hbox to 0pt{\hss chk (2)\hss}}%
    \hbox{\vrule depth0.500in width0pt height 0pt}%
    \kern 4.750in
  }%
}%

  \centerline{\raise 1em\box\graph}
  \vspace{2mm}
  \caption{Intersection Certificate Format (parenthesized numbers indicate byte counts)}
  \label{figure-certformats}
\end{figure}

Our certificate representation of an \RT\ credential contains the public
keys denoting entities mentioned in the credential. Roles are identified
by one byte numeric codes and are scoped by the entity defining the
role. Credential forms are distinguished by numbers $\{ 1, 2, 3, 4 \}$.
Certificates are also signed by their issuing authority. Conveniently,
the issuing authority is always mentioned in a credential, e.g.~the
issuing authority of $\cred{A.r}{B}$ is $A$, so the means to verify the
certificate (i.e.~the relevant public key) is always included with it
for free. This does not introduce a security problem. Since entities are
identified directly by their keys, an attacker who creates a new key is
simply creating a new entity.

The over-the-air format for the intersection certificate is illustrated
in \autoref{figure-certformats}. The other certificate forms are
organized in a similar way. The certificate starts with a certificate
type identifier byte, and then follows the written form of the \RT\
credential with a signature appended to the end. Role names are mapped
to single byte role numbers specified by the entity defining the role.

Certificates range in size from 124 bytes for the membership credential
to 166 bytes for the intersection credential. This is larger than the
maximum payload size limit of TinyOS T-Frame Active Message packets as
transported by IEEE 802.15.4 \cite{802.15.4,hui-tep125}. It is much
larger than the default maximum payload size of 28 bytes used by TinyOS
\cite{levis-tep111}. Consequently the certificates are fragmented into
four messages requiring a maximum payload size of 46 bytes. The choice
of using four fragments is a trade off between an excessive number of
fragments on one hand and excessive memory use for packet buffers on the
other. Also the session key negotiation messages described in
\autoref{section-session-key-negotiation} require 44 bytes in any case.

% These fragments are sent back to back with a 200 ms delay between each
% to allow the receiver time to assemble them. Fragments are sent in
% order with no fragment identifiers. To stay synchronized with the
% sender, receivers expect to receive all the fragments in a timely
% manner. If a fragment is not received within 750 ms of the previous
% fragment, the partial certificate is discarded on the assumption that
% the expected fragment was lost.

Verification of \RT\ certificates is the most computationally expensive
component of our system as we discuss in
\autoref{section-empirical-results}. Thus, we want to minimize the
amount of effort spent on verification. To this end, we append a 16-bit
Fletcher checksum to each certificate. Nodes maintain a database of
certificate Fletcher checksums to quickly check whether a certificate
has already been received and verified. It is necessary to include this
checksum rather than rely on the checksum provided by the underlying
network stack because the later value covers frame headers that are not
part of the certificate. Fletcher checksums are commonly used in WSNs
since their error detection properties are almost as good as CRCs with
significantly reduced computational cost \cite{fletcher-1982}.

% Currently certificates carry no lifetime information and are
% considered to be valid forever. This is not ideal since a certificate
% issuer may eventually change her policy but currently has no way to
% revoke old certificates. However, adding a feature for certificate
% revocation introduces non-monotonicity into the semantics of the
% authorization logic
% \cite{Li01nonmonotonicity,Rivest:1998:WEC:647502.728327}. Adding an
% expiration time to the certificates is more logically appealing but
% would require nodes to support real time services and some degree of
% time synchronization. This is a non-trivial extension of our basic
% system that is beyond the scope of our work.

\subsubsection{Session Key Negotiation}
\label{section-session-key-negotiation}

Public key cryptography is much too computationally expensive to use for
routine duty posting authorizations. Sprocket's run time system
addresses this by negotiating session keys between the sender (client)
and receiver (server) nodes.

%\autoref{figure-sessionkey-daemon} shows the session key
%processing architecture of a node.
%
%\begin{figure}[htbp]
%  \expandafter\ifx\csname graph\endcsname\relax
   \csname newbox\expandafter\endcsname\csname graph\endcsname
\fi
\ifx\graphtemp\undefined
  \csname newdimen\endcsname\graphtemp
\fi
\expandafter\setbox\csname graph\endcsname
 =\vtop{\vskip 0pt\hbox{%
    \special{pn 8}%
    \special{pa 0 700}%
    \special{pa 800 700}%
    \special{pa 800 300}%
    \special{pa 0 300}%
    \special{pa 0 700}%
    \special{fp}%
    \graphtemp=.5ex
    \advance\graphtemp by 0.500in
    \rlap{\kern 0.400in\lower\graphtemp\hbox to 0pt{\hss {\small Sender}\hss}}%
    \special{pa 1550 700}%
    \special{pa 2350 700}%
    \special{pa 2350 300}%
    \special{pa 1550 300}%
    \special{pa 1550 700}%
    \special{fp}%
    \graphtemp=.5ex
    \advance\graphtemp by 0.500in
    \rlap{\kern 1.950in\lower\graphtemp\hbox to 0pt{\hss {\small Receiver}\hss}}%
    \special{pa 495 1400}%
    \special{pa 1855 1400}%
    \special{pa 1855 1000}%
    \special{pa 495 1000}%
    \special{pa 495 1400}%
    \special{fp}%
    \graphtemp=.5ex
    \advance\graphtemp by 1.200in
    \rlap{\kern 1.175in\lower\graphtemp\hbox to 0pt{\hss {\small SessionKeyStorage}\hss}}%
    \special{sh 1.000}%
    \special{pn 1}%
    \special{pa 484 759}%
    \special{pa 400 700}%
    \special{pa 502 713}%
    \special{pa 484 759}%
    \special{fp}%
    \special{pn 8}%
    \special{pa 1175 1000}%
    \special{pa 493 736}%
    \special{fp}%
    \special{sh 1.000}%
    \special{pn 1}%
    \special{pa 1848 713}%
    \special{pa 1950 700}%
    \special{pa 1866 759}%
    \special{pa 1848 713}%
    \special{fp}%
    \special{pn 8}%
    \special{pa 1175 1000}%
    \special{pa 1857 736}%
    \special{fp}%
    \special{sh 1.000}%
    \special{pn 1}%
    \special{pa 275 100}%
    \special{pa 300 0}%
    \special{pa 325 100}%
    \special{pa 275 100}%
    \special{fp}%
    \special{pn 8}%
    \special{pa 300 300}%
    \special{pa 300 100}%
    \special{fp}%
    \special{sh 1.000}%
    \special{pn 1}%
    \special{pa 525 200}%
    \special{pa 500 300}%
    \special{pa 475 200}%
    \special{pa 525 200}%
    \special{fp}%
    \special{pn 8}%
    \special{pa 500 0}%
    \special{pa 500 200}%
    \special{fp}%
    \special{sh 1.000}%
    \special{pn 1}%
    \special{pa 1825 100}%
    \special{pa 1850 0}%
    \special{pa 1875 100}%
    \special{pa 1825 100}%
    \special{fp}%
    \special{pn 8}%
    \special{pa 1850 300}%
    \special{pa 1850 100}%
    \special{fp}%
    \special{sh 1.000}%
    \special{pn 1}%
    \special{pa 2075 200}%
    \special{pa 2050 300}%
    \special{pa 2025 200}%
    \special{pa 2075 200}%
    \special{fp}%
    \special{pn 8}%
    \special{pa 2050 0}%
    \special{pa 2050 200}%
    \special{fp}%
    \graphtemp=.5ex
    \advance\graphtemp by 0.150in
    \rlap{\kern 0.300in\lower\graphtemp\hbox to 0pt{\hss {\small Request\ }}}%
    \graphtemp=.5ex
    \advance\graphtemp by 0.150in
    \rlap{\kern 0.500in\lower\graphtemp\hbox to 0pt{{\small \ Reply}\hss}}%
    \graphtemp=.5ex
    \advance\graphtemp by 0.150in
    \rlap{\kern 1.850in\lower\graphtemp\hbox to 0pt{\hss {\small Reply\ }}}%
    \graphtemp=.5ex
    \advance\graphtemp by 0.150in
    \rlap{\kern 2.050in\lower\graphtemp\hbox to 0pt{{\small \ Request}\hss}}%
    \hbox{\vrule depth1.400in width0pt height 0pt}%
    \kern 2.350in
  }%
}%

%  \centerline{\raise 1em\box\graph}
%  \caption{Session Key Processing Architecture}
%  \label{figure-sessionkey-daemon}
%\end{figure}
%

The client maintains a session key storage that is indexed by the triple
$(N, C, I)$ where $N$ is the remote node ID, $C$ is the remote component
ID, and $I$ is the remote interface ID. A session key is thus created
for each combination of these IDs. The server also maintains a session
key storage indexed by $(N, C, I)$. In this case $N$ is the node ID of
the client and $C$, $I$ are the component and interface IDs on the
server to which that client is communicating. Since any given node can
be both server and client, each session key storage entry has a flag to
indicate the nature (client-side or server-side) of the session key.

% Because the session key storage is indexed by node ID and not by the
% requesting entity's key, this causes a (minor?) anomaly: If a client
% invokes a service ``as A'' and then later tries to invoke that same
% service ``as B'' the second invocation will use the session key
% created for the first. This is only a problem if A has access to the
% service but B does not. Note that it's not a problem for the server:
% the node is in domain A, with access, regardless. It's only a problem
% for a client that is trying to ``dumb down'' a particular request;
% access might succeed unintentionally.
%
% Fixing this would require storing some kind of key identifier (the
% whole key?) in the session key storage entry. Then duty post messages
% would need to include this identifier as well. This has non-trivial
% implications for memory and network overhead.

The first time a client attempts to access a service on a particular
server, it will send a session key negotiation request. When a server
receives a session key negotiation request message from a client node
$N$ containing the public key $K_{cp}$ of the requesting entity (as
mentioned in the \code{as} clause of the dynamic wire) and the $(C, I)$
address of the desired service, the following steps are taken:
\begin{longenum}
\item Authorization of $K_{cp}$ for service $(C, I)$ is checked using
  the \RT\ minimum model computed by the certificate receiver. If
  authorization fails nothing more is done.
\item A session key is computed using elliptic curve Diffie-Hellman and
  added to the session key storage under the proper $(N, C, I)$ value.
  The key is stored as a remote client key.
\item A message is returned to the client containing the server's public
  key $K_{sp}$ and the original $(N, C, I)$ values used by the client.
  This is so the client is able to compute the same session key and
  associate it with the proper endpoint from its perspective.
\end{longenum}

The session key negotiation protocol is a simple Diffie-Hellman key
agreement protocol that combines the public key of the peer entity with
the private key of the local entity. The implementation does not include
any nonces as would be done, for example, with the ECMQV protocol
\cite{ISO-IEC-1170-3:2008}. As a result any renegotiated session keys
between the same entities would be identical. However, this is not a
serious problem because the implementation does not currently
renegotiate session keys. Furthermore the ECMQV protocol entails three
exchanged messages and additional computations and so would further
increase the burden on nodes.

A potentially more serious concern is that the simple protocol described
here might be vulnerable to a man in the middle attack, whereby an
active attacker negotiates independent session keys with each peer and
is able to modify messages sent between those peers. However, in \RT\
private keys are identities directly so a man in the middle without
access to either the client or server's private key would appear as a
new entity, presumably without authorized access.

% Note that under this protocol every session key computed between nodes
% $N_A$ and $N_B$ for the same requesting entity will be the same. This
% is not a problem since the server node uses $(C, I)$ to look up the
% session key in its storage. If a node attempts to access an
% unauthorized service, no entry for that $(C, I)$ will exist in the
% server's session key storage and access will fail.

\subsubsection{Authorized RPC Invocations}

Authorized RPC invocations are made with MACs on invocation messages
using AES session keys. Verification of a MAC by the receiver
constitutes authorization since a session key for a particular client
and service is negotiated only after client credentials have been
verified as providing proof of compliance to authorization policy.
\autoref{figure-post} shows the format of authorized invocation request
messages. The interface and duty IDs, both four bit values, are packed
into the initial byte of the message. Following is the number of
receivers, the $(N, C)$ addresses of the receiving components, and the
duty arguments.

\begin{figure}[t]
  \expandafter\ifx\csname graph\endcsname\relax
   \csname newbox\expandafter\endcsname\csname graph\endcsname
\fi
\ifx\graphtemp\undefined
  \csname newdimen\endcsname\graphtemp
\fi
\expandafter\setbox\csname graph\endcsname
 =\vtop{\vskip 0pt\hbox{%
    \special{pn 8}%
    \special{pa 200 550}%
    \special{pa 450 550}%
    \special{pa 450 300}%
    \special{pa 200 300}%
    \special{pa 200 550}%
    \special{fp}%
    \special{pa 450 550}%
    \special{pa 700 550}%
    \special{pa 700 300}%
    \special{pa 450 300}%
    \special{pa 450 550}%
    \special{fp}%
    \graphtemp=.5ex
    \advance\graphtemp by 0.425in
    \rlap{\kern 0.575in\lower\graphtemp\hbox to 0pt{\hss {\small $n$}\hss}}%
    \special{pa 700 550}%
    \special{pa 950 550}%
    \special{pa 950 300}%
    \special{pa 700 300}%
    \special{pa 700 550}%
    \special{fp}%
    \special{pa 950 550}%
    \special{pa 1200 550}%
    \special{pa 1200 300}%
    \special{pa 950 300}%
    \special{pa 950 550}%
    \special{fp}%
    \special{pa 1950 550}%
    \special{pa 2700 550}%
    \special{pa 2700 300}%
    \special{pa 1950 300}%
    \special{pa 1950 550}%
    \special{fp}%
    \graphtemp=.5ex
    \advance\graphtemp by 0.425in
    \rlap{\kern 2.325in\lower\graphtemp\hbox to 0pt{\hss {\small args}\hss}}%
    \special{pa 2700 550}%
    \special{pa 3450 550}%
    \special{pa 3450 300}%
    \special{pa 2700 300}%
    \special{pa 2700 550}%
    \special{fp}%
    \graphtemp=.5ex
    \advance\graphtemp by 0.425in
    \rlap{\kern 3.075in\lower\graphtemp\hbox to 0pt{\hss {\small MAC}\hss}}%
    \special{pa 1200 300}%
    \special{pa 1950 300}%
    \special{dt 0.050}%
    \special{pa 1200 550}%
    \special{pa 1950 550}%
    \special{dt 0.050}%
    \special{pa 3450 300}%
    \special{pa 4200 300}%
    \special{dt 0.050}%
    \special{pa 3450 550}%
    \special{pa 4200 550}%
    \special{dt 0.050}%
    \special{pa 4200 300}%
    \special{pa 4200 550}%
    \special{dt 0.050}%
    \special{pa 325 300}%
    \special{pa 325 550}%
    \special{da 0.050}%
    \graphtemp=.5ex
    \advance\graphtemp by 0.300in
    \rlap{\kern 0.000in\lower\graphtemp\hbox to 0pt{\hss {\small Interface ID }}}%
    \special{sh 1.000}%
    \special{pn 1}%
    \special{pa 206 214}%
    \special{pa 263 300}%
    \special{pa 172 251}%
    \special{pa 206 214}%
    \special{fp}%
    \special{pn 8}%
    \special{pa 0 300}%
    \special{pa 100 150}%
    \special{pa 255 293}%
    \special{sp}%
    \graphtemp=.5ex
    \advance\graphtemp by 0.550in
    \rlap{\kern 0.000in\lower\graphtemp\hbox to 0pt{\hss {\small Duty ID }}}%
    \special{sh 1.000}%
    \special{pn 1}%
    \special{pa 295 595}%
    \special{pa 388 550}%
    \special{pa 327 634}%
    \special{pa 295 595}%
    \special{fp}%
    \special{pn 8}%
    \special{pa 0 550}%
    \special{pa 150 750}%
    \special{pa 380 556}%
    \special{sp}%
    \graphtemp=.5ex
    \advance\graphtemp by 0.000in
    \rlap{\kern 0.525in\lower\graphtemp\hbox to 0pt{\hss {\small Node ID}\hss}}%
    \special{sh 1.000}%
    \special{pn 1}%
    \special{pa 764 217}%
    \special{pa 825 300}%
    \special{pa 732 255}%
    \special{pa 764 217}%
    \special{fp}%
    \special{pn 8}%
    \special{pa 525 50}%
    \special{pa 748 236}%
    \special{fp}%
    \graphtemp=.5ex
    \advance\graphtemp by 0.000in
    \rlap{\kern 1.375in\lower\graphtemp\hbox to 0pt{\hss {\small Component ID}\hss}}%
    \special{sh 1.000}%
    \special{pn 1}%
    \special{pa 1168 255}%
    \special{pa 1075 300}%
    \special{pa 1136 217}%
    \special{pa 1168 255}%
    \special{fp}%
    \special{pn 8}%
    \special{pa 1375 50}%
    \special{pa 1152 236}%
    \special{fp}%
    \special{pa 700 650}%
    \special{pa 700 900}%
    \special{fp}%
    \special{pa 1950 650}%
    \special{pa 1950 900}%
    \special{fp}%
    \special{sh 1.000}%
    \special{pn 1}%
    \special{pa 800 800}%
    \special{pa 700 775}%
    \special{pa 800 750}%
    \special{pa 800 800}%
    \special{fp}%
    \special{sh 1.000}%
    \special{pa 1850 750}%
    \special{pa 1950 775}%
    \special{pa 1850 800}%
    \special{pa 1850 750}%
    \special{fp}%
    \special{pn 8}%
    \special{pa 800 775}%
    \special{pa 1850 775}%
    \special{fp}%
    \graphtemp=\baselineskip
    \multiply\graphtemp by 1
    \divide\graphtemp by 2
    \advance\graphtemp by .5ex
    \advance\graphtemp by 0.775in
    \rlap{\kern 1.325in\lower\graphtemp\hbox to 0pt{\hss {\small $n$ Components}\hss}}%
    \special{pa 2700 650}%
    \special{pa 2700 900}%
    \special{fp}%
    \special{pa 4200 650}%
    \special{pa 4200 900}%
    \special{fp}%
    \special{sh 1.000}%
    \special{pn 1}%
    \special{pa 2800 800}%
    \special{pa 2700 775}%
    \special{pa 2800 750}%
    \special{pa 2800 800}%
    \special{fp}%
    \special{sh 1.000}%
    \special{pa 4100 750}%
    \special{pa 4200 775}%
    \special{pa 4100 800}%
    \special{pa 4100 750}%
    \special{fp}%
    \special{pn 8}%
    \special{pa 2800 775}%
    \special{pa 4100 775}%
    \special{fp}%
    \graphtemp=\baselineskip
    \multiply\graphtemp by 1
    \divide\graphtemp by 2
    \advance\graphtemp by .5ex
    \advance\graphtemp by 0.775in
    \rlap{\kern 3.450in\lower\graphtemp\hbox to 0pt{\hss {\small $n$ Components}\hss}}%
    \hbox{\vrule depth0.900in width0pt height 0pt}%
    \kern 4.200in
  }%
}%

  \centerline{\raise 1em\box\graph}
  \caption{Duty Post Message}
  \label{figure-post}
\end{figure}

Since invocation of an RPC service on multiple hosts can be made at once
in a fan-out wiring (see \autoref{section-dynamic-wires}), a single
invocation request message may apply to multiple servers in the
neighborhood of the client. To conserve bandwidth, fan-out invocation
messages include multiple MACs, since separate session keys are
negotiated with each of $n$ servers, allowing a single message to invoke
the same service on the servers. If the duty arguments consume $d$ bytes
of data, then invocation messages consume $2 + 6n + d$ bytes. In the
common case where $n = 1$ invocation messages contain $8$ bytes of
overhead. As we describe above our implementation uses a 46~byte message
buffer size as required for sending certificate fragments. Our
experience suggests that using the same buffer size for invocation
messages allows for reasonable values of both $d$ and $n$.

% Alternatively an implementation could send multiple invocation
% messages with one for each server, reducing the number of MACs
% required on each message to one. However, that greatly increases radio
% traffic since the duty arguments and active message overhead must be
% duplicated for each message.

To conserve space in the invocation messages we only use a 32~bit MAC.
Such a small MAC would not normally be considered secure. However,
wireless sensor networks generate data so slowly that attacking even
such a short MAC is not considered feasible
\cite{karlog-tinysec-2004,luk-minisec-2007}.

\subsubsection{More on Security Properties}
\label{section-security-properties}

As in \autoref{section-security-model}, we stress that our scheme is
intended to enforce authorization, which we achieve via the protocols
described above. Integrity is a side effect of this, since we use MACs
to enforce authorization, which are computed over complete message
payloads and are verified by the receiver. Although confidentiality is
not directly supported by our current protocols, it could be easily
added. In particular payloads could be encrypted using negotiated
session keys (payloads are currently sent as plaintext).

Our system does not provide any form of replay protection out of the
box, but this can be added at the application level. For example an
application could pass a counter or time stamp as an additional duty
argument. The server could verify that the argument increases
monotonically as a simple form of replay protection. The space required
for counter or time stamp information would increase message sizes, but
this is unavoidable for any solution to replay protection. Delegating
replay protection to the application is appropriate since SpartanRPC is
intended to be a low level infrastructure on which more complex systems
can be built. Furthermore the need for replay protection is likely to be
application specific.

Perhaps the most problematic vulnerability of our system is to denial of
service (DoS) attacks-- unlike e.g.~replay, for which countermeasures
can be readily implemented at the application level, it is not clear how
DoS attacks could be mitigated without significant changes to the
underlying security protocols. For example, a constant flood of
certificates over the correct channel would place receiving nodes in a
constant state of signature verification, potentially consuming large
amounts of CPU time and energy. Mitigation of such attacks is outside
the scope of this work but has been discussed in the literature
\cite{4431860}.

\paragraph{A note on multicast security} Fan-out wirings are a common
idiom, and provide a form of multicast communication. However, the use
of MACs for security in a multicast setting presents well-known
challenges. In particular, while $n$-way Diffie-Hellman can be used to
negotiate secret keys between $n$ actors, such a scheme cannot be used
in light of the possibly heterogeneous authorization requirements we
anticipate. For example, suppose a node $A$ fan-out wires to service $s$
on distinct nodes $B$ and $C$, and suppose also that $A$ is authorized
for $s$ on both nodes but that $B$ is not authorized for $s$ on $C$ and
vice-versa. If a single session key were negotiated between $A$, $B$,
and $C$ in this case, then $B$ could make unauthorized use of $C$'s
version of $s$ and vice-versa. While a variety of techniques have been
proposed to mitigate this problem \cite{canetti-1999}, most typically
rely on very large multicast groups and are not applicable in our
setting. Thus, we handle fan-out wirings using multiple MACs as
described above.

\subsection{Identifying Services Over the Air}

RPC service endpoints are identified by the 4-tuple $(N, C, I, D)$ where
$N$ is the TinyOS ID of the node on which a duty is implemented. $C$ is
the local component ID assigned to each component that provides a
remotable interface. $I$ is an interface ID, required since a component
may provide more than one remotable interface. Interface IDs are
component-level unique. Finally $D$ is a duty ID, which must be
interface-level unique.

In the current version of \Sprocket, $(C, I)$ values are assigned
statically by an arbitrary (automated or social) process. \Sprocket\
accepts configuration files that define the association between $(C, I)$
values and the entities to which they refer. Duties are numbered in the
order in which they appear in their enclosing interface definitions.

Some RPC systems, such as ONC RPC, allow each node to provide a registry
of RPC services available on that node \cite{RFC-1833}. When a large
number of RPC services are provided by a node it becomes unreasonable to
expect clients to have hard coded knowledge of the endpoint identifiers
for all those services. Instead clients communicate with the single well
known registry to obtain endpoint identifiers that were dynamically
assigned. In contrast we assume this configuration information is known
a priori to all interacting actors. It is unclear if sensor networks
could benefit from a more sophisticated technique for defining and
communicating endpoint identifiers, but it would be an interesting topic
for future work.

\subsection{Rewriting SpartanRPC to nesC}

There are five major features requiring SpartanRPC-to-nesC rewriting by
\Sprocket: interface definitions, call sites where remote services are
invoked, duty definitions, dynamic wires, and server components
providing remote interfaces. In addition \Sprocket\ generates a stub
component for each dynamic wire, and a skeleton component for each
remote interface. Finally \Sprocket\ generates configurations that wrap
server components. Here we summarize rewriting strategies for these
features.

% The implementation doesn't actually wrap server components yet.
% Instead wiring to skeletons is done manually after Sprocket runs.
% Generating the wrapping components should be easy. Globally changing
% existing configurations to use the wrapping components will require
% some refactoring of Sprocket and would be more involved.

\subsubsection{Interfaces, Call Sites, and Duty Definitions}

Duty declarations in interfaces are rewritten to command declarations by
substituting \code{command} for \code{duty}. Since nesC commands are
allowed to have arbitrary parameters, duties with parameters present no
complications. \Sprocket\ verifies that if an interface contains a duty,
then the only declarations in that interface are duties. \Sprocket\
further verifies that the parameters of each duty, if any, conform to
the restrictions described in \autoref{section-remotable}.

% Some of these checks are not yet implemented, but they should present
% no problems. The current implementation only supports duties with
% parameters of primitive types (no structure types). A more significant
% limitation is that currently only interfaces with a single duty are
% supported. Generalizing to multiple duties in an interface should be
% straight forward.

Call sites where duties are posted are rewritten to command invocations
by substituting \code{call} for \code{post}. Only post operations
applied to duties are rewritten in this way. Finally, duty definitions
are rewritten to command definitions by also substituting \code{command}
for \code{duty}.

\subsubsection{Authorization Interfaces}

The rewriting process makes use of two interfaces that mediate the
interaction between the \Sprocket\ generated code and the security
processing components of the run-time system.
\autoref{figure-client-server-authorization} shows how a message,
entering from the left, is extended with authorization information by
the client and then passed to the server where the authorization
information is checked.

\begin{figure}[htbp]
  \expandafter\ifx\csname graph\endcsname\relax
   \csname newbox\expandafter\endcsname\csname graph\endcsname
\fi
\ifx\graphtemp\undefined
  \csname newdimen\endcsname\graphtemp
\fi
\expandafter\setbox\csname graph\endcsname
 =\vtop{\vskip 0pt\hbox{%
    \special{pn 8}%
    \special{pa 550 1400}%
    \special{pa 2150 1400}%
    \special{pa 2150 0}%
    \special{pa 550 0}%
    \special{pa 550 1400}%
    \special{fp}%
    \special{pa 2650 1400}%
    \special{pa 4250 1400}%
    \special{pa 4250 0}%
    \special{pa 2650 0}%
    \special{pa 2650 1400}%
    \special{fp}%
    \graphtemp=\baselineskip
    \multiply\graphtemp by -1
    \divide\graphtemp by 2
    \advance\graphtemp by .5ex
    \advance\graphtemp by 0.000in
    \rlap{\kern 0.550in\lower\graphtemp\hbox to 0pt{{\small Client Authorizer}\hss}}%
    \graphtemp=\baselineskip
    \multiply\graphtemp by -1
    \divide\graphtemp by 2
    \advance\graphtemp by .5ex
    \advance\graphtemp by 0.000in
    \rlap{\kern 2.650in\lower\graphtemp\hbox to 0pt{{\small Server Authorizer}\hss}}%
    \special{pa 700 1250}%
    \special{pa 2000 1250}%
    \special{pa 2000 750}%
    \special{pa 700 750}%
    \special{pa 700 1250}%
    \special{fp}%
    \graphtemp=\baselineskip
    \multiply\graphtemp by -1
    \divide\graphtemp by 2
    \advance\graphtemp by .5ex
    \advance\graphtemp by 1.000in
    \rlap{\kern 1.350in\lower\graphtemp\hbox to 0pt{\hss {\small AuthorizationClient}\hss}}%
    \graphtemp=\baselineskip
    \multiply\graphtemp by 1
    \divide\graphtemp by 2
    \advance\graphtemp by .5ex
    \advance\graphtemp by 1.000in
    \rlap{\kern 1.350in\lower\graphtemp\hbox to 0pt{\hss {\small (interface)}\hss}}%
    \special{pa 2800 1250}%
    \special{pa 4100 1250}%
    \special{pa 4100 750}%
    \special{pa 2800 750}%
    \special{pa 2800 1250}%
    \special{fp}%
    \graphtemp=\baselineskip
    \multiply\graphtemp by -1
    \divide\graphtemp by 2
    \advance\graphtemp by .5ex
    \advance\graphtemp by 1.000in
    \rlap{\kern 3.450in\lower\graphtemp\hbox to 0pt{\hss {\small AuthorizationServer}\hss}}%
    \graphtemp=\baselineskip
    \multiply\graphtemp by 1
    \divide\graphtemp by 2
    \advance\graphtemp by .5ex
    \advance\graphtemp by 1.000in
    \rlap{\kern 3.450in\lower\graphtemp\hbox to 0pt{\hss {\small (interface)}\hss}}%
    \special{pa 700 450}%
    \special{pa 1300 450}%
    \special{pa 1300 100}%
    \special{pa 700 100}%
    \special{pa 700 450}%
    \special{fp}%
    \graphtemp=.5ex
    \advance\graphtemp by 0.275in
    \rlap{\kern 1.000in\lower\graphtemp\hbox to 0pt{\hss {\small ACNullC}\hss}}%
    \special{pa 1400 450}%
    \special{pa 2000 450}%
    \special{pa 2000 100}%
    \special{pa 1400 100}%
    \special{pa 1400 450}%
    \special{fp}%
    \graphtemp=.5ex
    \advance\graphtemp by 0.275in
    \rlap{\kern 1.700in\lower\graphtemp\hbox to 0pt{\hss {\small ACRT0C}\hss}}%
    \special{pa 2800 450}%
    \special{pa 3400 450}%
    \special{pa 3400 100}%
    \special{pa 2800 100}%
    \special{pa 2800 450}%
    \special{fp}%
    \graphtemp=.5ex
    \advance\graphtemp by 0.275in
    \rlap{\kern 3.100in\lower\graphtemp\hbox to 0pt{\hss {\small ASNullC}\hss}}%
    \special{pa 3500 450}%
    \special{pa 4100 450}%
    \special{pa 4100 100}%
    \special{pa 3500 100}%
    \special{pa 3500 450}%
    \special{fp}%
    \graphtemp=.5ex
    \advance\graphtemp by 0.275in
    \rlap{\kern 3.800in\lower\graphtemp\hbox to 0pt{\hss {\small ASRT0C}\hss}}%
    \special{sh 1.000}%
    \special{pn 1}%
    \special{pa 1060 534}%
    \special{pa 1000 450}%
    \special{pa 1092 496}%
    \special{pa 1060 534}%
    \special{fp}%
    \special{pn 8}%
    \special{pa 1350 750}%
    \special{pa 1076 515}%
    \special{fp}%
    \special{sh 1.000}%
    \special{pn 1}%
    \special{pa 1608 496}%
    \special{pa 1700 450}%
    \special{pa 1640 534}%
    \special{pa 1608 496}%
    \special{fp}%
    \special{pn 8}%
    \special{pa 1350 750}%
    \special{pa 1624 515}%
    \special{fp}%
    \special{sh 1.000}%
    \special{pn 1}%
    \special{pa 3160 534}%
    \special{pa 3100 450}%
    \special{pa 3192 496}%
    \special{pa 3160 534}%
    \special{fp}%
    \special{pn 8}%
    \special{pa 3450 750}%
    \special{pa 3176 515}%
    \special{fp}%
    \special{sh 1.000}%
    \special{pn 1}%
    \special{pa 3708 496}%
    \special{pa 3800 450}%
    \special{pa 3740 534}%
    \special{pa 3708 496}%
    \special{fp}%
    \special{pn 8}%
    \special{pa 3450 750}%
    \special{pa 3724 515}%
    \special{fp}%
    \special{sh 1.000}%
    \special{pn 1}%
    \special{pa 2700 975}%
    \special{pa 2800 1000}%
    \special{pa 2700 1025}%
    \special{pa 2700 975}%
    \special{fp}%
    \special{pn 8}%
    \special{pa 2000 1000}%
    \special{pa 2700 1000}%
    \special{da 0.050}%
    \graphtemp=\baselineskip
    \multiply\graphtemp by -1
    \divide\graphtemp by 2
    \advance\graphtemp by .5ex
    \advance\graphtemp by 1.000in
    \rlap{\kern 2.400in\lower\graphtemp\hbox to 0pt{\hss Auth.\hss}}%
    \special{sh 1.000}%
    \special{pn 1}%
    \special{pa 600 975}%
    \special{pa 700 1000}%
    \special{pa 600 1025}%
    \special{pa 600 975}%
    \special{fp}%
    \special{pn 8}%
    \special{pa 0 1000}%
    \special{pa 600 1000}%
    \special{fp}%
    \graphtemp=\baselineskip
    \multiply\graphtemp by -1
    \divide\graphtemp by 2
    \advance\graphtemp by .5ex
    \advance\graphtemp by 1.000in
    \rlap{\kern 0.350in\lower\graphtemp\hbox to 0pt{\hss Message}}%
    \special{sh 1.000}%
    \special{pn 1}%
    \special{pa 4700 975}%
    \special{pa 4800 1000}%
    \special{pa 4700 1025}%
    \special{pa 4700 975}%
    \special{fp}%
    \special{pn 8}%
    \special{pa 4100 1000}%
    \special{pa 4700 1000}%
    \special{fp}%
    \graphtemp=\baselineskip
    \multiply\graphtemp by -1
    \divide\graphtemp by 2
    \advance\graphtemp by .5ex
    \advance\graphtemp by 1.000in
    \rlap{\kern 4.450in\lower\graphtemp\hbox to 0pt{Message\hss}}%
    \hbox{\vrule depth1.400in width0pt height 0pt}%
    \kern 4.800in
  }%
}%

  \centerline{\raise 1em\box\graph}
  \caption{Client/Server Authorization Architecture}
  \label{figure-client-server-authorization}
\end{figure}

The \texttt{AuthorizationClient} interface abstracts the details of how
an authorized message is prepared before being sent. The
\texttt{AuthorizationServer} interface abstracts the details of how
authorized messages are processed after they are received. This design
allows for pluggable authorization mechanisms. Future versions of
\Sprocket\ could potentially support other authorization schemes than
those described here, in a modular fashion.

The authorization interfaces provide their services in a split-phase
manner so that potentially long-running authorization computations can
be performed while allowing the node to continue other functions. In the
current implementation, two kinds of authorization are supported. On the
client side the precise method used depends on the dynamic wire over
which a particular communication takes place. On the server side it
depends on the presence of a \code{requires} clause on the remotely
provided interface.

The full $RT_0$ mechanism is supported by client and server components
\texttt{ACRT0C} and \texttt{ASRT0C} respectively (details about this
mechanism are discussed in \autoref{section-security-extensions} and
\autoref{section-security-protocols}). In addition a ``null''
authorization is supported by client and server components
\texttt{ACNullC} and \texttt{ASNullC} respectively. The null
authorization components perform no operation. They are used for dynamic
wires that do not require authorization and remote interfaces provided
publicly by servers.

\subsubsection{Dynamic Wires}

In the following, we use italics for metavariables that range over
arbitrary identifiers. The reader is referred to the rewriting schema
defined in \autoref{figure-dynamic-wire-rewriting}. Configurations
containing dynamic wires are rewritten to configurations that statically
wire the using component \code{\textit{ClientC}} to a stub
\code{Spkt\_\textit{n}} that interacts with the appropriate component
manager \code{\textit{SelectorC}} and that handles radio communication.
Every stub generated by \Sprocket\ is uniquely identified over the scope
of the entire program by an arbitrary integer $n$. The
\code{\textit{AuthorizerC}} component is \code{ACNullC} in the case
where no authorization is requested.

\begin{figure}[!t]
\begin{textbox}{3.0in}
\begin{Verbatim}[commandchars=\\\{\}, fontsize=\small]
\textrm{\textit{Dynamic Wire}}
    \textit{ClientC}.\textit{I} -> [\textit{SelectorC}].\textit{I};

\textrm{\textit{Rewritten as\ldots}}
    components Spkt_\textit{n};
    \textit{ClientC}.\textit{I} -> Spkt_\textit{n};
    Spkt_\textit{n}.ComponentManager -> \textit{SelectorC};
    Spkt_\textit{n}.AuthorizationClient -> \textit{AuthorizerC};
    Spkt_\textit{n}.Packet -> AMSenderC;
    Spkt_\textit{n}.AMSend -> AMSenderC;
\end{Verbatim}
\end{textbox}
\caption{Dynamic Wire Rewriting}
\label{figure-dynamic-wire-rewriting}
\end{figure}

In contrast a dynamic wire using either an \code{enable} or \code{as}
clause is rewritten the same way except that the
\code{\textit{AuthorizerC}} component is \code{ACRT0C}. Furthermore, the
list of enabled credentials is added to local certificate storage by
\Sprocket. Certificates in storage are periodically beaconed at run-time
as described above. Finally, the entity on whose behalf the RPC
invocation is performed is specified in the session key negotiation
message sent to the server, also as described above.

The \code{Spkt\_\textit{n}} stub provides the same interface provided by
\code{\textit{ClientC}}. Wherever a duty is posted by
\code{\textit{ClientC}} in source code, the rewritten call invokes code
in the stub that was specialized to handle that duty. The stub calls
into the component manager at run time to obtain a list of the dynamic
wire's endpoints and then prepares a data packet containing remote
endpoint addresses and marshaled duty arguments. Finally the stub calls
through the \code{AuthorizationClient} interface to perform whatever
authorization is needed.

\subsubsection{Remote Services}

For nodes supporting RPC services, \Sprocket\ generates a skeleton
component for each remote interface provided. This skeleton contains a
task corresponding to each duty provided in the interface, and every
generated skeleton is distinguished by a unique integer $n$ taken from
the same numbering space as the generated stubs.

% \autoref{figure-server-skeleton-generation} shows the form of a
% generated skeleton for an interface $I$ providing a single duty $d$
% that takes a single integer parameter. This is for purposes of
% illustration; the scheme is generalized in an obvious manner.

%\begin{figure}[!t]
%\begin{textbox}{3.0in}
%\begin{Verbatim}[commandchars=+\[\], fontsize=\small]
%+textrm[+textit[Server Component]]
%    module +textit[ServerC] {
%        provides remote interface +textit[I] requires +textit["A.g"];
%        +textit[other (local) uses/provides]
%    }
%
%+textrm[+textit[Skeleton generated as+ldots]]
%    module Spkt_+textit[n] {
%        uses interface +textit[I];
%        uses interface Receive;
%        uses interface AuthorizationServer;
%    }
%    implementation {
%
%        +textit[int value_1;]
%        +textit[task void d() {]
%            +textit[call I.d(value_1);]
%        +textit[}]
%
%        event message_t *Receive.receive( ... ) {
%            ...
%        }
%    }
%\end{Verbatim}
%\end{textbox}
%\caption{Server Skeleton Generation}
%\label{figure-server-skeleton-generation}
%\end{figure}

When messages are received on a node that provides RPC services, they
are examined to see if they are duty postings and thus to be handled by
a skeleton. If so, the \code{AuthorizationServer} interface is used to
authorize the message. If authorization succeeds, the task corresponding
to the specified duty is posted. That task simply calls into
\code{\textit{ServerC}} through the original interface
\code{\textit{I}}. Thus the task-like behavior of duties is ultimately
implemented using actual nesC tasks inside the server skeletons. Duty
parameters are conveyed via module-level variables accessed by the duty
tasks (since nesC tasks do not take formal arguments).

For each component that provides at least one remote interface,
\Sprocket\ creates a configuration that wires the corresponding
skeleton(s) to that component. This new configuration wraps the original
component and replaces uses of the original component in other
configurations in the program.

% In this Figure, as is the case for client-side code, the
% \code{\textit{AuthorizerC}} component is either \code{ASRT0C} or
% \code{ASNullC} depending on whether the original remote interface
% specified authorization or not.

% This is the part that isn't implemented yet. Right now skeleton wiring
% is done manually.

%\begin{figure}[!t]
%\begin{textbox}{3.0in}
%\begin{Verbatim}[commandchars=+\[\], fontsize=\small]
%configuration +textit[ServerC]_SpktC {
%    +textit[other (local) uses/provides]
%}
%implementation {
%    components +textit[ServerC], Spkt_+textit[n];
%    Spkt_+textit[n].+textit[I] -> +textit[ServerC];
%    Spkt_+textit[n].Receive -> AMReceiverC;
%    Spkt_+textit[n].AuthorizationServer -> +textit[AuthorizerC];
%    +textit[pass local uses/provides directly to ServerC]
%}
%\end{Verbatim}
%\end{textbox}
%\caption{Server Skeleton Wiring}
%\label{figure-server-skeleton-wiring}
%\end{figure}

