\section{Dynamic Wires}
\label{section-dynamic-wires}

In an ordinary nesC program the ``wiring'' between components as defined by configurations is
entirely static. The nesC compiler arranges for all connections and at run time the code invoked
by each called command or signaled event is predetermined.

In a remote procedure call system for wireless networks, this static arrangement is
insufficient. A node can not, in general, know its neighbors at compilation time but rather must
discover this information after deployment. In addition, the volatility of wireless links, and
of the nodes themselves, means that a given node's set of neighbors will change over time. In
this section we discuss the facility in SpartanRPC to allow \emph{dynamic wirings} for control
flow from duty invocation via remotable interfaces to duty implementation, wherein the
programmer has control over wiring endpoints and how they may change during program execution.

\subsection{Component IDs, Component Managers}
\label{section-componentmanager}

We begin by discussing how remote components are identified for wiring. In order to uniquely
identify components on the network, remotable components are specified via a two-element
structure called a \code{component\_id} defined in Fig.~\ref{figure-componentmanager}. The
\code{node\_id} member is the same node ID used by TinyOS and is set when the node is programmed
during deployment. The local ID member is an arbitrary value defined by the programmer of the
server node. Only components that are visible remotely need to have ID values assigned, however,
the ID values must be unique \emph{on the node}.
 
\begin{fpfig}[!t]{Component Manager Interface and Type Definitions}{figure-componentmanager}
{
\begin{Verbatim}
typedef struct {
    uint16_t node_id;
    uint8_t  local_id;
} component_id;

typedef struct {
    int count;
    component_id *ids;
} component_set;

interface ComponentManager { 
    command component_set elements();
}
\end{Verbatim}
}
\end{fpfig}

A \emph{component manager} is a component that provides the \code{ComponentManager} interface
defined in Fig.~\ref{figure-componentmanager}. It dynamically specifies a set of component IDs
that ultimately serve as dynamic wiring endpoints.

As a simple example, consider the component manager \code{RemoteSelectorC} as shown below:
\begin{Verbatim}
module RemoteSelectorC {
  provides interface ComponentManager;
}
implementation {
  component_id  broadcast  = {0xFFFF, 1};
  component_set remote_set = {1, &broadcast};

  command component_set
    ComponentManager.elements() {
        return result;
    }
}
\end{Verbatim}

This component manager always returns a component set containing a single component. The special
SpartanRPC broadcast node ID is used (\code{0xFFFF}) indicating that all neighbors should be the
target of the dynamic wire. The component ID on the neighbors is specified as 1 in this example.
In a more complex example the component manager would compute the component set each time the
dynamic wire is used, filling in an array of component IDs based on information gathered earlier
in the node's lifetime.

\subsection{Syntax and Semantics}

In SpartanRPC we extend the syntax and semantics of nesC to allow the target of a connection to
be dynamically specified by a component manager. The syntax of wirings, or connections, is
extended as follows:
\begin{Verbatim}
  connection ::=
      endpoint '->' dynamic_endpoint

  dynamic_endpoint ::=
      '[' IDENTIFIER ']' ('.' IDENTIFIER)*
\end{Verbatim}
\vspace{0.4em}

Given a dynamic wiring of the form \code{C.I -> [RC].I}, we informally summarize its semantics as
follows. First, we statically require that \code{RC} is a component manager, and that \code{I} is
remotable. At run time, if control flows across this wire via posting of some duty \code{I.d}
within \code{C}, the method \code{elements} in \code{RC} is invoked to obtain a set of component
IDs. The duties \code{I.d} provided by those remote components will then be posted on the host
machines via an underlying remote communication, the details of which are hidden from the
SpartanRPC programmer. Note that since this call to \code{elements} may return more than one
component ID, this is a sort of fan-out wiring.

For example, consider a simple service that allows client nodes to turn on or off three LEDs on
the server node. A client that wishes to use such a service could indicate its connection with
one or more server nodes using a configuration such as:
\begin{Verbatim}
ClientC.LEDControl ->
    [RemoteSelectorC].LEDControl;
\end{Verbatim}
\vspace{0.4em}

On the server the component that provides the LED controlling service must indicate that it is
to be provided remotely as shown in Sect.~\ref{section-duties-syntax}. The server's
configuration does not need to connect anything to the remote interface explicitly.

\subsection{Callbacks and First-Class IDs}

We assume that the local component IDs for well known services will be agreed upon ahead of time
by a social process outside of our system. By broadcasting to a well known local component ID, a
node can use services on neighboring nodes without necessarily knowing their node IDs.

If a node expects a reply from a service that it invokes, the calling node must set up a
component with a suitable remote interface to receive the service's result. In SpartanRPC remote
invocations can only transmit information in one direction. Bidirectional data flow requires
separate dynamic wires. In this case the service would normally require the client to provide
its component ID as an argument to the service invocation. The server could store that value for
use by a server-side component manager.

For example, assume that the LED controller on the server returns the old state of the LEDs
whenever the LED value is changed. The server configuration would include an appropriate dynamic
wire as follows
\begin{Verbatim}
LEDControllerC.LEDResult ->
    [LEDControllerC].LEDResult;
\end{Verbatim}

The client must provide the LEDResult interface remotely to receive this result. In this example
the \code{LEDControllerC} component is its own component manager. This makes it easy for the
\code{elements} command to access global data that was recorded inside \code{LEDControllerC}
when the service it provides was previously invoked. This is a common SpartanRPC idiom.
