\section{Introduction}
\label{section-intro}

SpartanRPC is an extension of the nesC programming language
\cite{Gay-nesC-2003} that supports application development in a
decentralized, open world security model for wireless sensor networks
(WSNs). Traditional networks have long enjoyed support for an open world
security model via public key based security architectures such as the
secure sockets layer (SSL). The goal of our work is to introduce an open
world security model to the TinyOS programming environment for embedded
device programming.

Currently, TinyOS security models are very simple and support only a
closed world paradigm. TinySec \cite{karlog-tinysec-2004} and MiniSec
\cite{luk-minisec-2007} are based on shared secrets and generally assume
that an entire network comprises a single security domain. Furthermore,
these systems support confidentiality and integrity properties, but not
access control, a.k.a. authorization. In contrast SpartanRPC includes
primitive features for specifying and enforcing authorization policies
and allows multiple security domains to interact within a single
network. SpartanRPC security mechanisms leverage public key cryptography
and an authorization logic to support an open world model where shared
secrets are not required a priori.

\subsection{Overview and Applications}
\label{section-overview}

The SpartanRPC system provides an applications programming interface for
managing resource access control in a WSN. It allows network
administrators to define security policies that mediate access to
specified resources on network nodes, and allows subnetworks from
different security domains to interact. A \emph{resource} in SpartanRPC
is user-defined functionality programmed in an extension of nesC, and
accessible via RPC by client code programmed in the same extension of
nesC. Thus, while previous systems have explored the problem of
establishing multiple security domains in a WSN
\cite{Claycomb:2011:NNL:1889383.1889450}, and other have considered RPC
in WSNs \cite{may-tinyrpc-2007}, SpartanRPC combines these features for
secure WSN application development in TinyOS. Furthermore, SpartanRPC's
expressive authorization logic allows specification of fine-grained and
decentralized security policies, better supporting collaborations
between multiple security domains.

Since the SpartanRPC API is flexible and easily accessible to TinyOS
programmers, these features can be readily used in a variety of
application spaces. Consider a first-responder situation, in which
multiple social entities must interact and cooperate on an ad-hoc basis.
Recent work has shown the effectiveness of WSNs in such scenarios
\cite{citeulike:4460555,1038146}, in their ability to coordinate
multiple data collection and communication devices in an easily
deployable manner. However, data collection and communication in this
scenario (and other similar ones) must be a secured resource, due to
e.g.~requirements of the Health Insurance Portability and Accountability
Act. Furthermore, security must be coordinated on-site in a WSN
comprising subnetworks administered separately (police, medical units
from different hospitals, etc.), and no prior coordination between
domains can generally be assumed. The SpartanRPC system is designed to
address these types of scenarios.

% For example, data communication can be treated as a secure resource by
% setting data access policies for individual nodes. WSN data
% communication protocols typically implement a \emph{publish/subscribe}
% semantics, whereby users of data ``subscribe'' to data produced by
% ``publishers''. Directed diffusion \cite{intanagonwiwat-2003} is one
% such protocol, where network nodes express \emph{interest} in certain
% types of data to neighboring nodes, which is subsequently reported to
% them when it is produced (published). In \autoref{section-example} we
% show how authorization policies can be assigned to interest update
% facilities on network nodes, which in effect imposes an access policy
% on data subscription in the network.

For example, if an emergency medical technician (EMT) emplaces a WSN to
monitor patient locations and vital signs, a security policy can be
imposed whereby responding police departments can emplace their own WSN,
and through it access patient identity and location data but \emph{not}
medical data directly from the EMT's network. This direct data access
will often be necessary due to real-time constraints and lack of
Internet connectivity in emergency situations.

Time synchronization is another important WSN function that is security
sensitive, since many higher-level protocols rely on it. A number of
previous authors have considered secure time synchronization in the
presence of ``insider'' attacks
\cite{Manzo:2005:TSA:1102219.1102238,Ganeriwal:2008:STS:1380564.1380571},
whereby nodes within the network may be compromised and function as
malicious actors capable of corrupting the protocol. In particular, the
FTSP protocol can be attacked by a single compromised ``root'' node
injecting false timing information into the network
\cite{Manzo:2005:TSA:1102219.1102238}, even when symmetric keys are used
for secure information exchange. However, the threat model in this work
treats all nodes in a network as equally compromisable. In cases where a
connected sub-network is more resistant to compromise, due to
e.g.~differences in hardware, a SpartanRPC policy can be established
whereby only nodes in the most tamper-resistant sub-component of the
network may function as roots. FTSP time sync updates on any given node
can be defined to require a root authorization level. This implies that
nodes requiring secure time synchronization must be at most a single
radio hop from a root node, but nodes willing to accept possibly
corrupted time sync data can extend the network indefinitely. Note that
in this scenario, SpartanRPC policies adapt to heterogeneity in network
device hardware, vs.~network administration as in the previous example.

Other potential applications of our system include secure routing
protocols in heterogeneous trust environments \cite{senroute-ahnj03},
transport and network layer protocols \cite{perillo-heinzelman-2005},
tracking protocols \cite{brooks-ramanathan-sayeed-2003}, and even
mote-based web servers supporting secure channels \cite{1049776}.  

As a concrete application example, in \autoref{section-snowcloud} we
describe our implementation of a WSN application for environmental data
sampling and collection. The node software provides remote resources for
data collection or node control. Access to these resources, in
particular via a distinguished sink node that is under control of some
user, is mediated by a security policy. Depending on the user's
credentials, only data collection, or collection and control, or
neither, will be allowed. Thus, we can specify that data end users can
be provided with sink nodes that only allow collection, whereas system
administrators can also control the network from theirs. And since user
populations may be part of different social entities than system
administration, our policy language allows authorization of a social
entity itself to administer its own user base for data collection. This
supports collaboration between different institutions which is a
hallmark of this particular application.

\subsection{Security Goals}
\label{section-security-model}

The main security-related goal of our work is to provide fine-grained
\emph{authorization} a.k.a. \emph{access control} in a WSN. We support
an \emph{open world} setting, where principals do not share secrets a
priori, and employ an authorization logic capable of expressing rich
security policies \cite{Li:DRBTMF}. As in various authorization
frameworks for more traditional distributed systems \cite{RFC-2693},
principals are identified as asymmetric key pairs, so that possession of
a private key is necessary and sufficient to establish a principal's
identity. With this understanding of identity, our underlying security
protocols support authentication of resource users through the use of a
Diffie-Hellman protocol. We assume that WSN nodes are provided with
private keys a priori by a trusted certification authority. Thus, we are
not concerned with communication of private keys.

Because WSN nodes are assumed to possess private keys, our system is
vulnerable to hardware tampering, which we consider out-of-scope for
this work. Modulo this, our system ensures that only authenticated
principals capable of proving compliance with a specified
authorization policy may access a protected resource. Authorization is
proved by communication of credentials to the authorizer in the form
of certificates, as discussed in
\autoref{section-security-extensions}. We note that the onus of
credential communication is on the requester, not the authorizer, who
only passively receives credentials. This implies that access may not
be granted in case incomplete credential sets are provided. Although
this incompleteness has been addressed in traditional distributed
systems via e.g.~distributed credential chain discovery
\cite{chapin-skalka-wang-acmcs08}, these approaches are relatively
heavyweight and difficult to implement in a WSN. The validity and
integrity of the certificates themselves are established via signature
verification by the authorizer.

In order to authenticate requesters, and to improve the efficiency of
normal message transfers between a requester and an authorizer,
symmetric session keys are computed using a simple Diffie-Hellman key
agreement protocol \cite{Diffie:2006:NDC:2263321.2269104} as described
in \autoref{section-session-key-negotiation}. The version of
Diffie-Hellman we use is not susceptible to man-in-the-middle attacks
because of the way principles are directly represented by their keys.
Thanks to this, and our certificate representation of credentials, the
SpartanRPC authorization scheme is not susceptible to man-in-the-middle
attacks, where an unauthorized principal (i.e.~a device not holding the
private key denoting an authorized principal) can obtain access to
resource. This claim is supported by relevant technical discussion in
\autoref{section-implementation}. We consider other security properties
of our system, relevant to DoS, replay, and confidentiality, in
\autoref{section-security-properties} following a more detailed
discussion of the language model implementation.

% We emphasize that the main security goal of the work presented here is
% an authorization mechanism able to support policies for the practical
% scenarios discussed in \autoref{section-overview}. In
% \autoref{section-snowcloud}, we demonstrate the effectiveness of our
% authorization framework in a relevant use-case.

\subsection{Technical Foundations}

Technical foundations underlying our secure RPC design include
programming language-based techniques, asynchronous RPC semantics, and
public key (PK) based authorization logics.

\paragraph{Language-Based Security} SpartanRPC provides language-level
abstractions for defining remote services and associated security
policies. Programmers are presented with an extension of nesC, with new
features for defining remote access controlled services, and for
invoking those services at specific authorization levels. This hides the
implementation details of underlying security protocols and only
requires mastery of a simple authorization logic. SpartanRPC programs
are compiled in the same manner as nesC programs, in fact the SpartanRPC
compiler rewrites SpartanRPC programs to nesC code and compiles the
latter.

\paragraph{Asynchronous Remote Procedure Call} As other authors have
observed \cite{may-tinyrpc-2007}, RPC is an appropriate abstraction for
node services on the network and supports whole-network
(vs.~node-specific) programming. Secure RPC is well-studied in a
traditional networking environment, and is a natural means of layering
security over a distributed communication abstraction.

It is necessary for RPC invocation in a WSN to be asynchronous, since
synchronous call-and-return to a remote node would significantly impede
performance in the best case and cause deadlock in the worst. In order
to minimally impact the nesC programming model, we define RPC invocation
as a form of \emph{remote task}. Local tasks are units of
programmer-defined asynchronous computation in nesC, so treating remote
computational services as remote tasks fits well in this paradigm.
Remote tasks can be invoked on one-hop neighbors, providing a link layer
service on which network layer services can be built. For example in
\autoref{section-example} we illustrate how a secure multi-hop data
collection protocol can be built using our link layer service.

\paragraph{PK-Based Authorization Policies} SpartanRPC provides
language-level abstractions for specifying RPC authorization policies.
The policy language we support is \RT\ \cite{Li:DRBTMF}, which allows
network entities to communicate composable credentials for authorizing
service invocations. Credentials are typically signed by certificate
authorities and do not require shared secrets to validate. SpartanRPC
certificates use elliptic curve cryptography (ECC) \cite{bertoni-2006}
signatures which are validated during an initial authorization phase.
ECC is significantly more tractable than RSA in a WSN setting. Following
the initial authorization phase our protocol establishes a shared AES
key for computing message authentication codes (MACs) during subsequent
invocations of a given service by the same node. Since hardware AES is
available on common WSN radio chipsets, we obtain highly efficient
performance for secure invocations following authorization. This is
demonstrated with empirical results reported in
\autoref{section-empirical-results}.

\subsection{Outline of Paper}

In \autoref{section-duties} we describe the fundamental language
abstraction we provide for defining remote services, called duties. In
\autoref{section-dynamic-wires} we describe a modification of remote
services that accommodates dynamically changing communication
neighborhoods. In \autoref{section-security-extensions} we define our
authorization logic, and show how to specify duty posting policies and
how they are enforced in the implementation. In
\autoref{section-example} we provide the extended example of secure
directed diffusion. In \autoref{section-implementation} we summarize
novel and important features of our implementation. In
\autoref{section-empirical-results} we discuss empirical results of
system performance. In \autoref{section-snowcloud} we describe a
realistic application of our system. We conclude with a discussion of
related work in \autoref{section-conclusion}.

% The benefits of this technology are manifold. Subnetworks from
% cooperating but distinct security domains could use each other's nodes
% to increase the resolution, spacial coverage, or lifespan of certain
% sensing or control functions. For example, assume that two distinct
% networks are deployed to the same space such that nodes from both
% networks can communicate with each other. While the primary purpose of
% the two networks would be independent, their administrators might
% nevertheless agree to collaborate on certain supporting functionality
% such as raw data collection, time synchronization, routing, fault
% tolerance, etc.

% Although in some cases data collected by each security domain could be
% shared off-network, perhaps via Internet connected gateways,
% advantages can be gained by enabling in-network interactions. For
% example if one network is partitioned due to a failed node, the
% separated partitions might be able to continue communicating by
% routing their messages over a cooperating network's nodes. In
% situations where low latency is important, such as with time
% synchronization protocols \cite{ganeriwal-kumar-srivastava-2003},
% in-network interactions between independent, cooperating networks
% would be more effective than attempting to synchronize two networks
% via long distance Internet links.

% Tracking applications \cite{brooks-ramanathan-sayeed-2003} could also
% benefit from in-network interactions between cooperating networks.
% Handing off tracking information from one network to another via an
% Internet link would require activating a potentially large number of
% nodes in each network in order to send information to their respective
% gateways. This creates power consumption concerns.

% Other potential applications of our system include secure routing
% protocols in heterogeneous trust environments \cite{senroute-ahnj03},
% transport and network layer protocols \cite{perillo-heinzelman-2005},
% tracking protocols \cite{brooks-ramanathan-sayeed-2003}, and even
% mote-based web servers supporting secure channels \cite{1049776}.

% High level applications can also benefit from in-network interactions
% between cooperating networks. For example, the use of WSNs to
% facilitate emergency care in disaster situations has been described
% \cite{citeulike:4460555,1038146}. While this previous work has focused
% on WSNs in a single security domain, it is likely that multiple
% domains would be in use during many emergencies as first responders
% from different organizations or different political jurisdictions work
% together. In those cases it is not realistic to assume that each
% domain would have Internet connectivity; the disaster site might be
% too remote or the supporting network infrastructure might be
% non-functional. Direct interaction between the WSNs of cooperating
% domains becomes an effective way for them to share information.
