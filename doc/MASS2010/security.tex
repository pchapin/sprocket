\section{Securing RPC}
\label{section-security-extensions}

Our primary goal is to provide convenient security primitives that application programmers can
use when developing heterogeneous wireless sensor networks. The RPC services presented in the
preceding sections provide an abstraction for building network services. Now, we focus on a
means to protect these services via a language-based authorization mechanism. This mechanism is
a simple \emph{capability-based} authorization system, that is sufficiently high-level to hide
the underlying message authentication scheme, while being sufficiently low-level to serve as a
basis for more complex protocols in applications.

\subsection{Capability-Based Security}

A capability is an unforgeable reference to a resource, the possession of which is necessary to
gain resource access \cite{884422}. In SpartanRPC resources are taken to be RPC services, and
programmers may specify security policies associated with these services by requiring the
activation of a capability for their usage. We envision that individual capabilities will
be associated with statically assigned roles.

\subsection{Syntax and Semantics}

To declare security policy, an RPC service provider can modify a remote interface provides
declaration with the syntax $\code{requires}\ C$ where $C$ is a string literal denoting a
capability. For example:
\begin{Verbatim}
module LEDControllerC {
    provides remote interface LEDControl
        requires "K";
}
\end{Verbatim}
This means that posting of any duty in \code{LEDControl} provided by this component requires
activation of capability \code{K}. All capability requirements are declared statically in this
manner, and all capabilities are given statically, i.e.~SpartanRPC does not support dynamic
capability generation. This is reasonable because capabilities are associated with network
services. Thus the number of needed capabilities scales with the number of interacting security
domains and not with the total number of nodes.

In order to use a secured service, clients may activate a capability $C$ when wiring to a
protected component interface via the syntax $\code{auth}\ C$. For example, assuming that
$\code{RemoteSelectorC}$ is a component manager for provider(s) of the protected \code{LEDControl}
service described above:
\begin{Verbatim}
auth "K" ClientC.LEDControl ->
    [RemoteSelectorC].LEDControl;
\end{Verbatim}
\vspace{0.4em}
Any postings of duties from \code{LEDControl} in the \code{ClientC} component need not mention
security at all---capabilities are activated at configuration wiring connections since that is
where interface uses are reified with implementations.

\subsection{Security Properties}

The implementation of our security mechanism is described in more detail in
Sect.~\ref{section-implementation}, but a summary is as follows. Capabilities are in one-to-one
correspondence with AES symmetric keys. Activating a capability $C$ at a wiring connection
entails signing all messages associated with duty postings over that wire with a MAC using the
key denoted by $C$. Any service provider will verify these MACs under the key denoted by the
capability $C$ required for the service. Since capabilities are known statically, we assume that
keys are stored in ROM, so that a node's capabilities are exactly the keys it is deployed with.

Our system does not provide any form of replay protection out of the box, but this can be added
at the application level. For example an application could pass a counter as an additional duty
parameter. The server could verify that the count increases monotonically as a simple form of
replay protection.

We feel that delegating replay protection to the application is appropriate since SpartanRPC is
intended to be a low level infrastructure on which more complex systems can be built. Not all
applications will need replay protection and it is our desire to keep the core overhead minimal.

In addition our system does not currently offer any confidentiality service. However, extending
our system to encrypt duty arguments is an area we intend to explore as future work.
