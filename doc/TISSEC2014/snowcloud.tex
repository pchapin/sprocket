\section{A Prototype Application}
\label{section-snowcloud}

To evaluate the performance of SpartanRPC in a real application
setting, we have used the system to implement secure versions of data
collection and sampling control protocols in an environmental
monitoring system. The Snowcloud system
\cite{frolik-skalka-snowcloudtr,moeser-walker-skalka-frolik-wsc11} is
a WSN developed at the University of Vermont for snow hydrology
research applications.  It is based on the MEMSIC TelosB mote platform
running TinyOS, and has seen multiple field deployments. Typical
deployed systems comprise 4-8 sensor nodes but the technology is
currently scalable to arbitrary numbers of nodes. For data collection
and sampling rate control, the system also includes a handheld
``Harvester'' device.  This device incorporates a TelosB mote to
establish a network connection when in radio communication with the
deployment.  Users transport the device to and from deployment sites,
and interact with the sensor node network by issuing commands from a
simple push-button interface. A Harvester device and a deployed
Snowcloud sensor tower are pictured in \autoref{figure-snowcloud}. The
scheme described here has been implemented and tested in our test
network at UVM, which uses the same software and hardware platforms as
in our active deployments.

\snowcloudfig

In our secured version of the Snowcloud system, the goal is to treat
data collection and sampling rate control as protected resources
requiring authorization. Furthermore, sampling rate modifications
should require a higher, ``administrator'' level of authorization than
data collection. That is, only system engineers should be able to
perform control operations, whereas data end-users making field visits
should be able to collect data.  Snowcloud sensor node code in
particular makes use of nearly every resource available on the mote--
including timing, sensor I/O, radio messaging, and flash memory, not
to mention CPU and main memory. Thus, it is a robust example of the
interaction of SpartanRPC with mote resources in real applications.

The system described here is also informative since it can be easily
ported to other similar application settings. That is, WSN application
settings wherein multiple users of various authorization levels need
to interact with the same network in control or collection capacities,
as mediated by security policy. The SpartanRPC API allows
straightforward retasking of authorized service implementations to
these various settings. Furthermore, the RT authorization logic
supports collaboration between multiple social domains, by allowing
security policy to be managed in a decentralized manner as we
illustrate below.

\subsection{Security Policies}

To specify and implement the security policies informally described
above, we consider the sensor network and the Harvester single node
``network'' as separate security domains, each with their own set of
credentials. The sensor network is always endowed with
administrator-level credentials. If a Harvester is to be used by a
system engineer, its mote is also endowed with administrator-level
credentials, whereas a Harvester to be used by a data end-user is only
endowed with user-level credentials. When a Harvester is introduced to
the sensor network, its resource accesses are mediated by its
authorization level. Since credentials are unforgeable, a user-level
Harvester can never be used for sensor network control even if it is
reprogrammed.

Sensor nodes within the network possess four credentials, as follows.
In these credentials the Snowcloud domain is abbreviated
$\mathit{SC}$. Authority to collect data and control sensors in the
network are governed by the roles $\mathit{SC.Col}$ and
$\mathit{SC.Con}$, respectively.  Credential (1) says that control
authority contains collection authority. (2) says that nodes in the
Snowcloud domain have control authority. (3) says that any entity in a
Snowcloud collaborator's $\mathit{Usr}$ role has collection
authority. (4) says that the node identified by $\mathit{NId}$ is in
the Snowcloud domain.
\begin{mathpar}
(1)\quad \cred{SC.Col}{SC.Con}

(2)\quad \cred{SC.Con}{SC.Node}

(3)\quad \cred{SC.Col}{SC.Collab.Usr}

(4)\quad \cred{SC.Node}{NId}
\end{mathpar}
When invoking remote services, the node will do so on behalf of the
entity $\mathit{NId}$. It will also be imaged with the $\mathit{NId}$
private key for session key negotiation.

Any Harvester within the Snowcloud domain is then provided with the
credential $\cred{SC.Node}{HId}$ and the $\mathit{HId}$ private key
issued by Snowcloud domain administration. This will provide that
Harvester with collection and control authority in the domain. For
Harvesters to be provided to collaborators, the Snowcloud
administrators issue a credential establishing the institution as a
collaborator, while the institution itself may define and manage policy
for their $\mathit{Usr}$ role membership. For example, the University
of New Hampshire ($\mathit{UNH}$) can be established as a collaborator
with credential (5) issued by Snowcloud domain administration, and may
specify role membership with the credential (6) issued by UNH domain
administration:
\begin{mathpar}
(5)\quad \cred{SC.Collab}{{UNH}}

(6)\quad \cred{{UNH}.Usr}{UsrID}
\end{mathpar}
These two credentials, along with the $\mathit{UsrID}$ private key,
are imaged on Harvesters issued to UNH collaborators for data
collection, but which remain unauthorized for control.

\subsection{Implementation}

Resources themselves are accessed through a secure command
dissemination protocol, that is modeled upon the TinyOS Dissemination
protocol (as described in TEP 118). In short, protected RPC services
establish network level broadcast channels requiring authorization for
use. Commands are communicated to the network over these channels, and
different channels are used for different sorts of commands. 

In more detail, command broadcast services can be specified as a duty
in a remotable interface:
\begin{Verbatim}[fontsize=\small]
    interface SpDisseminationUpdate { duty void change(command_t new_value); }
\end{Verbatim}
To implement e.g.~the control command channel, the following module
can be defined and included on sensor nodes in the Snowcloud domain:
\begin{Verbatim}[fontsize=\small]
    module ControlDissemC {
        provides remote interface SpDisseminationUpdate requires "SC.Con";
        uses            interface SpDisseminationUpdate as NeighborUpdate;
        provides        interface ComponentManager;
    }
    implementation { ... }
\end{Verbatim}
In the implementation, the provided $\verb+SpDisseminationUpdate+$
interface accepts command invocations from neighbors, but requires
them to be authorized for the $\mathit{SC.Con}$ role. Commands are
relayed to all other neighbors (i.e.~disseminated) via the used
$\verb+NeighborUpdate+$ interface; those neighbors are identified by
the provided $\verb+ComponentManager+$.

To use this component, both sensor and Harvester nodes can configure 
it through the following component instantiation and wiring, where 
the component's $\verb+NeighborUpdate+$ interface is wired remotely
to neighbors: 
\begin{Verbatim}[fontsize=\small]
    components ControlDissemC as ControlChan;
    activate "*" for 
        ControlChan.NeighborUpdate -> [ControlChan].SpDisseminationUpdate;
\end{Verbatim}
Note that a node must be endowed with the appropriate credentials for
this wiring to be useful. 

This same code pattern can be used to implement a data collection
request channel, protected by the $\mathit{SC.Col}$ role instead of
$\mathit{SC.Con}$. In response to an authorized control command
invocation, sensor nodes will modify their behavior appropriately,
whereas in response to authorized data collection requests sensor
nodes will report their data using collection tree protocol (TEP 123)
to the Harvester.

Note that since the only ``border'' between security domains in this
scenario is between the Harvester and its neighbor(s), Snowcloud
scalability is not affected. Only authorization between the Harvester
and its one-hop neighbors needs to be established no matter what the
network size, and since areal coverage is the goal of a deployment,
network densities remain fairly constant where neighborhoods are on
the order of 1-5 nodes in conceivable deployments.

\subsection{Results}

Results can be characterized according to both the application
end-user experience and to quantitative aspects. As detailed in
\autoref{section-empirical-results}, a one-time transient overhead is
imposed for initial credential exchange and session key negotiation
when a Harvester is first introduced to the network. However, since
data collection for a network after several months of deployment can
take up to an hour, this overhead is relatively insignificant. And
steady-state overhead is small, and does not affect data collection
rates. Further, subsequent field visits will not impose transient
overhead since negotiated keys can be cached in non-volatile
memory. Thus, authorized user experience is not significantly impacted
by the addition of security.

From a quantitative perspective, the most important measurements to
consider for this application, beyond the general ones already
considered in \autoref{section-empirical-results}, are RAM and ROM
consumption of the insecure and secured versions of the Harvester
collection protocol. We have to consider whether layering SpartanRPC
security over a realistic application will overrun the resources
available to a mote platform. Relevant measurements are as follows.
\begin{table}[h]
\centering \newcommand\T{\rule{0pt}{2.1ex}}
  \tbl{RAM and ROM use for Snowcloud versions}
  {
  \begin{tabular}{|l|c|c|} \hline
    \emph{Program}       \T & \emph{RAM Bytes} & \emph{ROM Bytes} \\ \hline\hline
    Insecure Harvester   \T & 2274             & 24316 \\ \hline
    Secure Harvester     \T & 4771             & 35834 \\ \hline
    Insecure Sensor Node \T & 2868             & 36254 \\ \hline
    Secure Sensor Node   \T & 5417             & 48616 \\ \hline
  \end{tabular}
  }
  \label{table-snowcloud}
\end{table}

Both RAM and ROM consumption are significantly increased by the addition
of SpartanRPC security to this application. However, these numbers are
within operating parameters, and the Sprocket implementation of
SpartanRPC described in \autoref{section-implementation} has not yet
been optimized for space efficiency in any way; improvements in this
respect can be made but are out of scope for this work.
