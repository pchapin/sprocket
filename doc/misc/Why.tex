%%%%%%%%%%%%%%%%%%%%%%%%%%%%%%%%%%%%%%%%%%%%%%%%%%%%%%%%%%%%%%%%%%%%%%%%%%%%
% FILE    : why.tex
% AUTHOR  : Peter C. Chapin <pchapin@cems.uvm.edu>
% SUBJECT : A few thoughts on RPC based security in WSNs.
%
%%%%%%%%%%%%%%%%%%%%%%%%%%%%%%%%%%%%%%%%%%%%%%%%%%%%%%%%%%%%%%%%%%%%%%%%%%%%

%+++++++++++++++++++++++++++++++++
% Preamble and global declarations
%+++++++++++++++++++++++++++++++++
\documentclass{article}
%\documentclass[12pt]{article}

% Common packages. Uncomment as needed.
%\usepackage{amsmath}
\usepackage{amssymb}
%\usepackage{amstext}
%\usepackage{amsthm}
%\usepackage{doublespace}
%\usepackage{fullpage}
%\usepackage[dvips]{graphics}
%\usepackage{listings}
%\usepackage{mathpartir}
%\usepackage{url}
%\usepackage{hyperref}

% The following are settings for the listings package.
%\lstset{language=Java,
%        basicstyle=\small,
%        stringstyle=\ttfamily,
%        commentstyle=\ttfamily,
%        xleftmargin=0.25in,
%        showstringspaces=false}

\setlength{\parindent}{0em}
\setlength{\parskip}{1.75ex plus0.5ex minus0.5ex}

% Some handy macros.
\newcommand{\todo}[1]{\textit{TODO: #1}}
\newcommand{\po}{\preccurlyeq}

%++++++++++++++++++++
% The document itself
%++++++++++++++++++++
\begin{document}

%-----------------------
% Title page information
%-----------------------
\title{Motivations}
\author{Peter C. Chapin}
\date{March 27, 2010}
\maketitle

\section{Introduction}

Currently most wireless sensor network (WSN) applications are envisioned as single networks that
are under the control of a single administrative domain. This domain includes all the network
nodes as well as the base station(s) and any other devices that might request the services of
the network. Consequently security concerns in WSNs have been largely focused on securing the
network and its endpoints from external agents that have no legitimate business with the
network.

My work is of a different nature. I am concerned with the case where multiple security domains
are involved. This can occur when there is more than one WSN operating in the same physical
space so that nodes from multiple networks can interact directly. This can also occur when
network endpoints (base stations or computationally powerful mobile devices) from one security
domain enter the field of a another domain's network and wish to access that network.

The systems I have been working with are designed to allow in-network security processing in
these cases. However, one might reasonably ask why this is necessary. If each network is
connected to a powerful machine that can act as a security gateway, then why can't networks from
multiple domains interact via their gateways? Such an approach off loads the inter-domain
security processing to machines with rich computational resources.

The purpose of this white paper is to explore these issues in detail.

\section{Definition of Terms}

In this white paper I will often talk about wireless sensor networks in different security
domains. For my purposes I will define a security domain as a collection of resources (often
computational resources) that are managed and controlled by a single policy maker. I will
normally assume that all machines in the same domain trust each other for some appropriate
definition of trust.

Let the nodes of a single domain wireless sensor network be defined by the pair $(N, E)$ where
$N$ is the set of all nodes participating in mutual radio communication. The set $E \subseteq N$
is the set of \emph{endpoint} nodes. An endpoint node is a computationally powerful system that
serves as an external interface to the WSN. Many WSNs have fixed base stations as endpoints, but
in some cases a hand held mobile unit can serve in this role. An endpoint might be directly
associated with a human user or it might be a gateway to some other system such as the Internet.
Notice that according to this definition the nodes in $E$ are in the same security domain as the
nodes in $N$.

Although I have defined a network's nodes $A = (N, E)$ it is convenient in many cases to refer
to $A$ as if it was the set $N$ directly. That is, I will speak of a node $a \in A$ when I
really mean that $a \in N_A$.

Let $C_{AB}$ be the \emph{inter-domain connection relation from network $A$ to network $B$}. A
pair of nodes $(a, b)$, with $a \in A$ and $b \in B$, are in this relation if and only if $b$
can directly (that is, using a single hop) receive signals from $a$. Notice that because of
asymmetries due to differences in transmitter power output, receiver sensitivity, or non-linear
propagation effects, it is not the case that $(a, b) \in C_{AB} \rightarrow (b, a) \in C_{BA}$

Network $A$ is \emph{connected} to network $B$ if and only if $(C_{AB} \ne \emptyset) \vee
(C_{BA} \ne \emptyset)$. Informally two networks are connected if it is possible for
communication to pass between them in at least one direction. Note that if $A$ is connected to
$B$ then $B$ is connected to $A$ by this definition.

Network $A$ is \emph{overlapping} network $B$ if and only if $(C_{AB} \ne \emptyset) \wedge
(C_{BA} \ne \emptyset)$. Informally two networks are overlapping if bi-directional communication
between the networks is possible.

The \emph{degree of overlap} between $A$ and $B$ is $d_{AB} \equiv \min(|C_{AB}|, |C_{BA}|)$.
Notice that if the networks are not overlapping the degree of overlap is zero. Also notice that
if $A$ contains $n$ nodes and $B$ contains $m$ nodes, the maximum possible degree of overlap is
$nm$. This occurs when every node in $A$ can communicate with every node in $B$ (in only one
hop) and visa-versa.

\section{Basic Scenario}

To understand the fundamental issues associated with multiple domain WSNs consider a simple
application where two overlapping networks $A$ and $B$ agree to share temperature data.
Specifically temperature information at each node is periodically sent back to $A$'s base
station. Here I assume that networks $A$ and $B$ have some private functionality that is not
shared as well (otherwise why not just have a single security domain?).

Sharing temperature data in the network can be avoided if $A$ and $B$ simply gather their own
data independently and then share that information between Internet connected base stations.
However, by working together in the field, complete data can potentially still be collected even
if one of the networks is partitioned due to node failure. For example if $A$'s network is
partitioned, the partition that would normally be inaccessible from $A$'s base station may be
able to continue sending data through $B$'s network.

What about power consumption? Assume that $A$ and $B$ compute an inter-domain routing tree
rooted at $A$'s base station over which temperature data is sent. This routing tree would be
independent of any internal routing tree created by $A$ and $B$ in support of their private
functions. Assuming the temperature data can be aggregated in the network (for example, perhaps
only the maximum temperature is needed), then each node must send exactly one message. This
entails $n + m - 1$ total messages\footnote{$n = |N_A|$ and $m = |N_B|$}. In the case of
independent networks a total of $n + m - 2$ messages are needed. Thus it would seem that there
would be no power savings if the two networks worked together in the field.

In fact, if the data can not be aggregated in the network, the two independent networks would
probably save energy since the depth of the two routing trees for $A$ and $B$ alone would
probably be less than that of the combined routine tree. \todo{Do the math.}

However, if the degree of overlap is large, the combined routing tree would probably involve
shorter hops on average than the individual routing trees. This could save energy by allowing
the nodes to transmit with reduced power. \todo{Do the math. This could be an important point.}


\section{Low Latency Scenario}

Gathering temperature data independently and then sharing it over the network is fine. However,
some algorithms have stringent latency requirements that could make the use of an Internet link
difficult. One example is time synchronization algorithms. Suppose networks $A$ and $B$ agree to
work together to synchronize time on their nodes. This might be done to increase the liklihood
of a node finding a suitable partner with which to synchronize. However, the delays due to
involving the Internet in such a protocol might be unacceptable. Nodes in $A$'s network need to
interact directly with nodes in $B$'s network \emph{in the field} to execute such an algorithm
successfully.


\section{Cooperative Scenario}

Some WSN algorithms require cooperation between adjacent nodes in the field. Algorithms for
tracking objects, for example, need to wake up adjacent nodes and hand off tracking
responsibility to them. Communication with adjacent nodes in another domain will be much more
efficient (in terms of power consumption) if that communication is direct rather than via
Internet connected endpoints that may be many hops away.


\section{Free Floating Scenario}

So far I've assumed that all WSNs have at least one Internet connected endpoint that can be used
for communication with other WSNs. This might not be true in remote areas where Internet
connectivity is expensive, but such areas are unlikely to have overlapping WSNs from multiple
security domains.

There is, however one important scenario to consider: that of the \emph{free floating network}.
Such a network has no Internet connected endpoints, or even no endpoints at all. Instead it
exists entirely to service requests from mobile devices \emph{from other security domains} that
overlap with it. Consider for example a wireless network set up to monitor road conditions. A
vehicle outfitted with an appropriate transceiver could interact with this network to give the
driver up-to-minute information about poor road conditions ahead. While it might be
theoretically possible for the on-board computers in the vehicle to communicate with the road
network via the Internet, this might also be infeasible in many cases.

Another example of a free floating network might be a battlefield WSN that gathers data on troop
movement. Soldiers from the side deploying the network would like to access its information
while in the field without granting access to that information by the enemy. In a combat
situation it is unreasonable to assume that Internet connectivity would exist.

Note that in the scenarios just described the WSN plays the role of a resource server only (as
far as the mobile device is concerned). One could imagine the WSN playing the role of an RPC
client as well... for example automatically detecting and using resources on the mobile devices
as they move into the field of a WSN.

\bibliographystyle{plain}
\bibliography{references}

\end{document}
