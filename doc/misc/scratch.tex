\section{Introduction}

\subsection{Motivation: the Need for Security}

\subsection{WSN RPC is Fundamental}

We believe that RPC is the right fundamental abstraction for defining control security in WSN
applications. RPC will allow nodes to provide ``libraries'' of services to external
applications, without the need for reprogramming. This latter point is important since in a
heterogeneous trust environment, we expect that only node owners will be allowed to reprogram
them. Indeed, we consider node reprogramming security outside the scope of our investigation.

Rather than build an RPC mechanism from scratch, we plan to leverage existing technology. One
previous system for ``anycast RPC'' is interesting, and provides a rich variety of RPC services
\cite{bergstrom-pandey-mass07}. However, this approach is too heavyweight for our goal of
providing fundamental features to support higher-level abstractions. Also, their system is
designed for OS*, not nesC and TinyOS, whereas we aim to develop technology for the latter
platform since currently it is prevalent. A more attractive option for our purposes is TinyRPC
\cite{may-etal-mass05}, which is lighter weight and designed as an extension to nesC
\cite{781133}.

The language nesC is a component-based variant of C endowed with a component wiring language.
More deeply, it employs a \emph{split-phase} semantics where function call-and-return is broken
into two parts; a \emph{command} phase where functionality is invoked, and an \emph{event
  handler} phase where callbacks are implemented. These phases can also be one-way, in the sense
that a command need not be accompanied by a terminating event that must be handled, and events
can be signaled independently of any preceding command. For a summary of nesC, we refer the
reader to \cite{781133}.

While TinyRPC adds powerful node communication features to nesC programming, it makes only a
simple extension to the source language. Specifically, wiring configurations may refer to
\texttt{remote} interfaces. For example, the wiring \texttt{M.I -> remote I[NODE\_ID, N]} will
wire the local interface \texttt{I} used by the module \texttt{M} to the interface \texttt{I}
provided by the module \texttt{MODULE\_ID} on node \texttt{NODE\_ID}, the node identifier being
either statically defined or determined by a dynamic binding mechanism provided by the
implementation. The details of wiring and node communication are handled transparently by the
TinyRPC implementation. Since commands and events are invoked exclusively via interfaces in
nesC, this remote wiring specification is sufficient to support the RPC abstraction. In the
following, we hide some details of TinyRPC syntax for the sake of simplicity.

\section{Data and Control Security}

\section{Security Model}

Choosing an appropriate security model is crucial to our system design. As usual for WSNs,
efficiency is a central consideration. While modern authorization systems such as RT
\cite{Li:DRBTMF} and SPKI/SDSI \cite{RFC-2693} provide a rich policy specification logic,
computing authorization decisions at the node level in such systems is much too expensive to be
feasible. Although rich policy specification logics may be useful at a higher, network
programming level, we propose to compile such specifications down to a simpler underlying
authorization system, as will be discussed in \autoref{section-macroprogramming}. Our goal in
the design of SpartanRPC is to provide a low-level authorization mechanism that is sufficiently
expressive to support higher-level specifications, but is still efficient enough for practical
use in WSNs.

To this end we propose a simple role based access control (RBAC) model \cite{Sandhu:RBACM}. In
RBAC, roles are uniquely identified, and are associated with particular resources;
\emph{membership} in a role is a necessary and sufficient condition to access those resources.

In our system roles are arranged in a lattice structure along with an ordering relation $\po$.
Informally $R_1 \po R_2$ means that role $R_1$ is less privileged than role $R_2$. However, the
precise meaning and significance of the roles is left to the application to decide. For example,
an application that is interested in integrity could define $R_1 \po R_2$ to mean that role
$R_1$ is more trusted than role $R_2$. The bottom role $\bot$ corresponds to the case where no
role is activated. Roles are combined using the usual least upper bound $\wedge$ and greatest
lower bound $\vee$ operations. For efficiency reasons on a single role is active at any time.
This prevents our system from dealing with an open ended set of roles during authorization
computations.

In order to obtain access to a resource, the requester must prove the relevant role membership.
More expressive systems employ more complicated proof procedures, but typically proof of role
membership is supplied via some collection of certificates \cite{Clarke:CCDSS,Li:DCDTM}. In our
system, roles will be comparatively simple (i.e.~no ``attributes'' in the typical RBAC sense),
and identified with particular symmetric cryptographic keys. Role membership will be proved by
key signing.

This scheme has a number of benefits. First, efficient symmetric key cryptography has been
developed for WSNs \cite{rogaway-ocb-2003,1031515}, so this approach is realistic. And as will
be discussed more in \autoref{section-spartanrpc-implementation}, the same symmetric key
infrastructure can establish both data and control security, simplifying our implementation. In
a more theoretical spirit, basing access control on certified role membership, rather than node
identities, supports a data-centric communications model. And as will be discussed more in
\autoref{section-macroprogramming}, this simple and highly efficient RBAC model can serve as an
offline compilation target for more expressive policy languages.

We assume that roles are defined statically and are not treated as first class values, and can't
be passed as function or event parameters. This means that new keys aren't generated
dynamically, and key distribution is handled at the same time as program distribution. This
simplifies our security model, and will allow us to treat key distribution statically, using
existing efficient systems for this purpose
\cite{Eschenauer-keymanagement-2002,Huang-keymanagement-2004,du-keydistribution-2005,liu-keydistribution-2005,camtepe-bulent-05}.
In this presentation we will use the security roles \texttt{Low}, \texttt{Medium}, and
\texttt{High} to illustrate ideas, but instances of our system are free to name their own roles.

\section{Language-Based Security Features} 

\subsection{Specifying Control Security} 

The split phase computational model of TinyOS is appropriate when control flows from a higher
level region into a lower level region. High level code invokes functionality in low level code
by issuing a command and receives the results from the code by way of an interrupt-like event.
However, the nodes in a WSN are normally peers with neither node being clearly ``higher level''
than the other. In addition, wireless communication is time consuming and unreliable. For these
reasons SpartanRPC makes use of a task like model of remote procedure call. In SpartanRPC remote
code is presented as \emph{duties} that are executed asynchronously and that produce no return
value directly.

In programming language-based security, an important issue is \emph{where} in programs policy
should be specified. In SpartanRPC programs, two natural options are to allow policy annotations
on interfaces where duties are specified, or on modules where they are implemented. Specifying
security on interfaces is in keeping with modern software design principles \cite{503253};
however, in a nesC setting, multiple modules may implement the same interface, so policy
specifications on interfaces would constrain every module implementing a given interface to
exhibit the same security policy. We therefore advocate policy specification in modules to allow
greater flexibility in this regard. This approach is similar to Java stack inspection, where
code is annotated with policies via privilege checks \cite{pottier-skalka-smith-toplas05}. And
as we will discuss below, a component of our system will be a compile time type analysis that is
able to \emph{infer} the security policies induced on components by function-level policy
declarations and program control flow. Thus, induced security policies can be ``advertised''
along with component specifications.

For example, consider the following interface \texttt{TempSampleControl}, specifying a duty
\texttt{set\_threshold} that allows the threshold of temperature sensing to be set. A companion
interface \texttt{TempSample} specifies a duty that will receive the temperature information.
\begin{verbatim}
interface TempSampleControl {
  duty void set_threshold(int threshold);
}

interface TempSample {
  duty void temperature(int value);
}
\end{verbatim}

A node that \texttt{provides TempSampleControl} may wish to protect the \texttt{set\_threshold}
duty, since invocation will alter node operation. In our system, when implementing a module that
\texttt{provides TempSampleControl}, the programmer can require that any poster of
\texttt{set\_threshold} must be able to assume a \texttt{High} security role in the command
code, via the \texttt{requires} command name modifier:
\begin{verbatim}
module TemperatureSensor { 
  provides TempSampleControl;
  ...
  duty void TempSample.set_threshold(int threshold)
    requires "High" { ... }
}
\end{verbatim}

A node with a temperature sensor may wish to restrict the class of users who can control its
temperature threshold. However, a node that receives temperature results may also wish to ensure
that the sensor is properly authenticated as well.
\begin{verbatim}
module TemperatureUser {
  provides TempSample;
  ...
  duty void TempSample.temperature(int value)
    requires "Medium" { ... }
}
\end{verbatim}

% TODO: I need to figure out how tasks are to be handled.
More generally the definition of a function, command, event, or duty can be decorated with a
\texttt{requires} clause that specifies the role that is must be active before that function,
command, event, or duty can be executed. More precisely we define the notion of current role
$R_c$. During execution the role specified on a definition $R$ must obey the relation $R \po
R_c$.

For local execution the requirements on role activation are ensured by way of a static control
flow analysis. Roughly this analysis proceeds as follows.
\begin{enumerate}
\item The role requirement on each function $R_I$ is inferred from the functions that it calls.
  The overall requirement is the least upper bound on all the roles that are required by the
  functions that are called. However functions that are called inside a role activation block
  (described below) are not considered. Role activation blocks are used to change roles based on
  application defined criteria. For example a function that requires a low privilege might
  activate a highly privileged role after validating the calling context to perform a sensitive
  operation on behalf of the caller.
\item If a function has a required role $R$ explicitly declared then it is a compile time
  requirement that $R_I \po R$. That is, the required role must be at least as powerful as that
  inferred by the function's code.
\item Note that command and event invocations cross (static) wires and the control flow analysis
  must use information in the configuration components to properly ascertain which functions are
  actually invoked.
\end{enumerate}

Because of this static analysis no run time checking of roles is needed for code that executes
entirely locally.

\subsection{Using Secure Remote Modules} 

So how does the poster of a duty assume a role in order to access a secured resource? At the
language level, we provide a basic abstraction that is inspired by the Java stack inspection
security model. Programmers will be able to specify ``secure regions'' in code via an
\texttt{activate} construct\footnote{The use of ``activate'' is intended to evoke the notion of
  RBAC role activation.}. For example, any code that \texttt{uses} a \texttt{TempSampleControl}
interface with the expectation that it will be (dynamically) wired to a providing component can
post the \texttt{set\_threshold} duty as follows:
\begin{verbatim}
   activate("High") { post TempSampleControl.set_threshold(...); }
\end{verbatim}
Similarly, to send a \texttt{temperature} event to the handler:
\begin{verbatim}
   activate("Medium") { post TempSample.temperature(...); }
\end{verbatim}
Any amount of code can be nested in a \texttt{activate} block. In the implementation of
SpartanRPC, the details of which we discuss in \autoref{section-spartanrpc-implementation}, the
identifiers \texttt{High} and \texttt{Medium} are transparently linked to cryptographic keys,
and are semantically meaningful only if those keys are possessed by the node executing the
secure region. The above post operations will be performed with these keys, and verified on the
receiver for authorization.

This approach has two distinct advantages. First, code will not run in a privileged state unless
specified by the programmer, in keeping with the principle of least privilege. Second, modules
can be defined that invoke secure duty postings, but which are \emph{not} wrapped in
\texttt{activate} blocks. This will allow the definition of library code that makes use of
secure remote services in an opaque manner, but which does not automatically authorize the use
of remote services, or even require the presence of keys associated with roles at compile time.
Of course, this library code itself must be wrapped in a \texttt{activate} block to be executed.

\subsection{Examples}

\section{Type Analysis}

\subsection{Static Type Analysis}

\subsection{RPC Type Analysis}

\section{Applications: Secure Directed Diffusion}
