\section{Example: Secure Directed Diffusion}
\label{section-example}

To illustrate our language design we have implemented a secured version
of the well-known directed diffusion protocol \cite{intanagonwiwat-2003}
for ad-hoc routing of data in a sensor network. It is one example of the
publish/subscribe paradigm for data gathering. In our secure version
facilities for subscribing to a data stream are defined as secure RPC
services by data stream providers. Directed diffusion supports multi-hop
data collection, so this example illustrates how our link layer RPC
service supports network layer communication. It also serves as a good
benchmark application for empirical observations reported in
\autoref{section-empirical-results}. We provide another extended example
in a real prototype application of SpartanRPC in
\autoref{section-snowcloud}.

The directed diffusion algorithm \cite{intanagonwiwat-2003} allows a
node to subscribe to a data stream by expressing an \emph{interest} in
it. In our example, an interest is expressed as temperature data above a
given threshold. A certain data rate, expressed as a time interval
between samples, is associated with each interest. Initially a node
seeking temperature data floods the network using an interest with a low
data rate. As data events find their way back to the interested node,
that node selectively \emph{reinforces} certain immediate neighbors by
retransmitting the interest with a higher associated data rate to just
those neighbors. Also, in our version of directed diffusion we imagine
that it is to be implemented in a network comprising multiple security
domains, and specify that subscription to data streams requires certain
authorization levels as defined by policy. We omit the policy
specification in this example to focus on the language API. An example
policy specification is presented and discussed in
\autoref{section-snowcloud}.

\begin{comment}
  \subsection{Directed Diffusion in Multiple Domains}

  To demonstrate our authorization logic, we imagine that subscription
  to a data stream requires a certain authorization level. Assume that
  the Federal Emergency Management Agency (FEMA) manages a role
  \code{tempControllers} that specifies entities with the authority to
  define who can use the temperature sensing service. FEMA does not own
  any sensor networks itself but is nevertheless trusted to make
  appropriate delegation decisions, and so can function as a kind of
  certificate authority.

  FEMA can delegate control over the service to the state of Vermont
  \code{V} by issuing a certificate denoting the following credential:
  \begin{displaymath}
    \code{FEMA.tempControllers} \leftarrow \code{V}
  \end{displaymath}

  Note this certificate will be signed by FEMA. The state of Vermont can
  then place fire departments in various jurisdictions into a
  \code{tempUsers} role by issuing, for example, its own signed
  certificates denoting the following credentials:

  \begin{mathpar}
    \code{V.tempUsers} \leftarrow  \code{A}

    \code{V.tempUsers} \leftarrow  \code{B}
  \end{mathpar}

  where \code{A} is the \RT\ entity of the Addison town fire department
  and \code{B} is the \RT\ entity of the Burlington city fire
  department.

  The sensor network of the Addison town fire department protects access
  to the temperature sensing application using the governing role
  \code{A.t}. The nodes are deployed with the following credential
  pre-loaded to serve as access control policy.

  \begin{displaymath}
    \code{A.t} \leftarrow \code{FEMA.tempControllers.tempUsers}
  \end{displaymath}

  This policy allows access to an entity \code{E} if a FEMA
  \code{tempController} says \code{E} is a \code{tempUser}. The Addison
  town fire department nodes are also deployed with the following
  certificates gathered off-line from public sources such as FEMA's web
  site.

  \begin{mathpar}
    \code{FEMA.tempControllers} \leftarrow \code{V}

    \code{V.tempUsers} \leftarrow \code{A}
  \end{mathpar}

  These certificates prove that \code{A} is a \code{tempUser} according
  to at least one FEMA certified \code{tempController}. Similarly the
  Burlington city fire department protects its temperature sensing
  application using the governing role \code{B.t}. Its nodes are
  deployed with a similar policy statement and with certificates showing
  that \code{B} is a member of \code{V.tempUsers}.

  If both fire departments are called upon to work together fighting a
  large fire, their sensor networks will now be able to interoperate,
  for purposes of temperature sensing, \emph{with no prior
    coordination.} Furthermore suppose the fire rages out of control and
  the fire department from neighboring Crown Point, New York, \code{C}
  is called in to assist. The new nodes introduced by \code{C} will also
  be able to interoperate with both sets of existing nodes provided they
  are deployed with certificates such as the following, where \code{N}
  is the \RT\ entity of the state of New York:

  \begin{mathpar}
    \code{FEMA.tempControllers} \leftarrow \code{N}

    \code{N.tempUsers} \leftarrow \code{C}
  \end{mathpar}

  In our example we implemented the program for the Addison town fire
  department. However, the other programs are identical, from the point
  of view of temperature sensing, except for changes in the governing
  role on the remote services and changes in the deployed policy and
  certificates.
\end{comment}

\subsection{Interfaces}

Interest and data propagation are handled by separate interfaces, as
shown in \autoref{figure-ddinterfaces}, each containing a single duty.

\begin{fpfig}[!t]{Directed Diffusion Interest and Data Management Interfaces}{figure-ddinterfaces}
{
\vspace{0.6em}
\begin{Verbatim}[fontsize=\small]
    interface InterestManagement {
        duty void set_interest(uint16_t sender_node, int temp_threshold, 
                               int interval, int duration);
    }

    interface DataManagement {
        duty void set_data(uint16_t sender_node, uint16_t originator_node, 
                           int temp_value);
    }
\end{Verbatim}
\vspace{0.3em}
}
\end{fpfig}

A node expresses interest in temperature data above a certain threshold
and at a certain data rate by posting the \code{set\_interest} duty on
its neighboring nodes. The duration parameter of the
\code{set\_interest} duty specifies a lifetime of the interest. Once an
interest expires it is removed from the node's interest cache.
Temperature values are expressed as integers presumably corresponding to
the output of an analog to digital converter. Similarly time intervals
are expressed as integer multiples of some unit time, the exact value of
which is arbitrary.

A node passes data to its interested neighbors by posting the
\code{set\_data} duty on those neighbors. The \code{originator\_node} is
the ID of the node where the data was originally observed; the node
soliciting this data will typically want to know its provenance.

Both of these interfaces include the sender's node ID as an explicit
parameter. The nodes use that information to track paths through the
network. Although the node ids are also part of the low level radio
packets sent between the nodes, we chose for demonstration purposes to
manage the interest and data information strictly at a higher level in
the protocol stack. As usual greater efficiency may be possible by
mixing protocol layers.

\subsection{Configuration}

The interest and data caches, which we call ``managers,'' are the two
central components of our application. The interest manager provides the
\code{Interest\-Management} interface remotely and uses the same
interface on other components. The data manager provides and uses the
\code{Data\-Management} interface in a similar way. Both components
serve as their own component managers, using internal information to
specify the destination nodes of each outgoing post operation.

The main configuration contains, in part, the following wiring for the
interest manager:

\begin{lrbox}{\savebigbox}
\begin{minipage}{4.5in}
\vspace{0.6em}
\begin{Verbatim}[fontsize=\small]
enable "*" for InterestManagerC.NeighborSensors ->
                  [InterestManagerC].InterestManagement;
\end{Verbatim}
\vspace{0.3em}
\end{minipage}
\end{lrbox}
\centerline{\usebox{\savebigbox}}

In this example, we assume the nodes are deployed with a single default
entity as their identity. As discussed in
\autoref{section-rpc-client-side}, because the \code{as} clause is
missing this wiring makes the invocation on behalf of that entity. We
anticipate that single entity nodes will be common.

Because the interest manager provides and uses the same interface, it
defines \code{NeighborSensors} as an alias for the
\code{Interest\-Management} interface that it uses remotely. When the
interest manager posts the \code{set\_interest} duty, that duty is
invoked in all neighbors \emph{currently} selected by its own, internal
component manager. These post operations are authorized using
credentials available to the invoking node; neighbors can be in multiple
security domains.

\subsection{Authorized Interest Management}

The interest manager has a partial specification as follows, which we
assume resides in a domain controlled by some \texttt{Admin} entity.
Observe that its \texttt{InterestManagement} interface requires
authorization for the \texttt{Admin.OK} role:

\begin{lrbox}{\savebigbox}
\begin{minipage}{4.9in}
\vspace{0.6em}
\begin{Verbatim}[fontsize=\small]
module InterestManagerC {
    provides interface ComponentManager;
    provides remote interface InterestManagement requires "Admin.OK";
    uses interface InterestManagement as NeighborSensors;
}
\end{Verbatim}
\vspace{0.3em}
\end{minipage}
\end{lrbox}
\centerline{\usebox{\savebigbox}}

Because the interest manager is its own component manager, setting up
target node addresses entails updating an internal
\code{component\_set}. In the case when a new interest is received the
interest manager propagates that interest to all neighbors. This is done
inside the interest manager's \code{set\_interest} duty as shown in
\autoref{figure-interest-propagation}.

\begin{figure}[!t]
\begin{textbox}{3.25in}
\begin{Verbatim}[fontsize=\small]
remote_set.ids   = &remote_components;
remote_set.count = 1;
remote_components[0].node_id  = 0xFFFF;
remote_components[0].local_id = INTEREST_ID;
post NeighborSensors.set_interest( ... );
\end{Verbatim}
\end{textbox}
\caption{Propagation of New Interests}
\label{figure-interest-propagation}
\end{figure}

The ``well known'' local component ID of the interest manager is used to
specify which component on neighbor nodes is to process the duty. The
implementation of the \code{elements} command in the
\code{Component\-Manager} interface merely returns \code{remote\_set}
computed above. Before the posting of \code{set\_interest} returns,
\code{remote\_set} is used to prepare the outgoing packet. After the
post is complete \code{remote\_set} and \code{remote\_components} can be
reused without affecting any pending radio transmissions.

% There is an issue with the current implementation and broadcasts. The
% session key storage doesn't know how to initiate session key
% negotiation with multiple neighbors (with unknown IDs) at once. This
% issue is solvable. The runtime system needs to maintain a list of
% neighboring nodes. Whenever the SpartanRPC broadcast address is used,
% the runtime system would do a multicast to the neighbor list instead.
% The neighbor list is discovered and maintained by yet another
% background process. It should be a small list so memory consumption
% should be minimal.

In the more complicated case where an interest is being reinforced, the
interest manager must use information in the data cache to compute which
neighbors need reinforcing. Although SpartanRPC allows a component
manager to dynamically select neighbor nodes, the component used as a
component manager is statically bound. Thus in this example the interest
manager cannot switch its component manager to, for example, the data
manager. To work around this, the interest manager communicates with the
data manager using connections not shown here. With the data manager's
help the interest manager computes appropriate neighbors dynamically
before posting \code{set\_interest} on those neighbors. The data manager
implementation has a similar structure and authorization mechanism.

\begin{comment}
  \subsection{Data Management}

  The data manager has a dual structure where the implementation of the
  \code{set\_data} duty simply adds the data event to the data cache,
  and the implementation of a timer fired event performs the task of
  propagating data to interested nodes. The data manager manipulates the
  timer frequency to match the highest required data rate. However since
  not all data needs to be sent to all neighbors at such a high rate,
  only a dynamically changing subset of neighbors is selected for each
  timer event. This is done by adjusting an internal
  \code{component\_set} before posting the \code{set\_data} duty.

  We further assume that nodes will only want to accept data events from
  authorized producers. All legitimate posts of \code{set\_data} must be
  done using the same role as for interests. The specification of the
  data manager provides interface \code{DataManagement} requiring role
  \code{Admin.OK} but there are no security related artifacts in the
  body of the data manager's implementation.
\end{comment}
