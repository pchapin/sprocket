\section{Introduction}
\label{section-intro}

As WSN technology becomes more ubiquitous applications of overlapping yet independently
controlled sensor networks will begin to appear. Networks from cooperating but distinct security
domains may wish to use each other's nodes to increase the resolution, spacial coverage, or
lifespan of certain sensing or control functions. For example, assume that two distinct networks
are deployed to the same space such that nodes from both networks can communicate with each
other. While the primary purpose of the two networks might be independent, their administrators
might nevertheless agree to collaborate on certain supporting functionality such as data
collection and analysis.

Although in some cases data collected by each security domain could be shared off-network,
perhaps via Internet connected gateways, certain advantages could be gained by enabling
in-network interactions. For example if one network is partitioned due to a failed node, the
separated partitions might be able to continue communicating by routing their messages over a
cooperating network's nodes.

In situations where low latency is important, such as with time synchronization protocols
\cite{ganeriwal-kumar-srivastava-2003}, in-network interactions between independent, cooperating
WSNs would be more effective than attempting to synchronize two networks via long distance
Internet links. Tracking applications \cite{brooks-ramanathan-sayeed-2003} could also benefit
from in-network interactions between cooperating networks. Handing off tracking information from
one network to another via an Internet link would require activating a potentially large number
of nodes in each network in order to send information to their respective gateways. This creates
power consumption concerns. Other potential applications where in-network interactions could
play a role include secure routing protocols in heterogeneous trust environments
\cite{senroute-ahnj03}, transport and network layer protocols \cite{perillo-heinzelman-2005},
and even mote-based web servers supporting secure channels \cite{1049776}.

High level applications can also benefit from in-network interactions between cooperating WSNs.
For example, the use of WSNs to facilitate emergency care in disaster situations has been
described \cite{citeulike:4460555,1038146}. While this previous work has focused on WSNs in a
single security domain, it is likely that multiple domains would be in use during many
emergencies as first responders from different organizations or different political
jurisdictions work together. In those cases it is not realistic to assume that each domain would
have Internet connectivity; the disaster site might be too remote or the supporting network
infrastructure might be non-functional. Direct interaction between the WSNs of cooperating
domains becomes an effective way for them to share information.

In this paper we describe novel middleware technology to support WSN applications in this
setting. Our system, which we call SpartanRPC, provides a new form of link-layer RPC as a
natural extension of the nesC programming model, and \emph{language based authorization} based
on symmetric-key cryptography. As other authors have observed \cite{may-tinyrpc-2007}, RPC is an
appropriate abstraction for node services on the network and supports whole-network
(vs.~node-specific) programming. We further observe that RPC allows nodes to provide flexible,
modular services without the need for reprogramming, a kind of ``micro web services.''
\emph{Secure} RPC is also clearly desirable in a heterogeneous trust environment.

\subsection{Overview and Contributions}

Previous related work illustrates interest in and useful applications of RPC in a WSN context.
For example, the Marionette system uses network layer RPC for remote (PC-based) analysis and
debugging of WSNs \cite{whitehouse-marionette-2006}. The Fleck operating system provides a small
pre-defined set of RPC services for WSN applications, while the secFleck system extends this
with a form a secure RPC \cite{hu-secfleck-2009}. SpartanRPC differs from these systems in that
it extends the nesC programming language to allow programmer definition (unlike secFleck) of
secure RPC services that can be accessed by nodes within the network itself (unlike Marionette).
Our system is similar to and inspired by TinyRPC \cite{may-tinyrpc-2007}, except the latter does
not provide security and has a different semantics that is not as expressive and flexible as our
approach.

Major contributions of our work include an RPC design that is consistent with existing nesC
semantics, including an asynchronous ``task-like'' conception of RPC and \emph{dynamic wires} as
a natural extension of configuration wirings to allow flexibility in remote communication. We
also provide a mechanism for fine-grained RPC authorization. Finally we created an
implementation of SpartanRPC \cite{sprocket} from which we obtained empirical results showing
that SpartanRPC features are not onerous in terms of additional space and energy consumption.
   
A primary goal of the SpartanRPC design is to provide RPC capabilities as transparently as
possible. This means at least that the low-level communication details should be hidden from the
user. Beyond that, it means that RPC features should be provided in a manner that fits in with
existing nesC semantics.

\paragraph{Asynchronous Execution Model} In nesC a \emph{command} is a synchronous unit of
control; when one is called, it is pushed onto the call stack and the current continuation waits
for the command to complete and yield a result. In contrast, when a \emph{task} is posted it is
placed in a queue for execution at some future time, and the calling context continues
immediately. We propose a task-like mechanism for RPC invocation, that allows \emph{remote
  postings}; in this our approach differs from TinyRPC that envisioned RPC as a form of command
\cite{may-tinyrpc-2007}. However, our mechanism differs from tasks in two important ways: first,
we allow arguments to be passed in RPC calls, and second, the module that posts an ordinary task
must define it, whereas we want to allow modules to execute functionality defined non-locally.
We therefore introduce the \emph{duty} mechanism. Duties may be posted like a task, but passed
arguments and provided by modules for posting by other (possibly non-local) modules. Duties are
discussed in Sect.~\ref{section-duties}.

\paragraph{Dynamic Wires} The SpartanRPC design provides a minimalist extension of configuration
wiring syntax with abstractions for remote communication. In our system components can provide
remotable interfaces, and in configuration definitions these interfaces can be wired to locally
or non-locally. However, wirings to remote interfaces must specify the host of the wired-to
component, and our syntax requires the user to provide an inherently dynamic definition of
endpoint hosts to allow fan-out wirings that can change at run-time. Since SpartanRPC works at
the link layer, we believe this flexibility is fundamental in a general WSN setting where
neighborhoods can be expected to change frequently due to node repositioning, failure, or
changes in radio signal strengths. Dynamic wires are discussed more in
Sect.~\ref{section-dynamic-wires}.

\paragraph{Fine-Grained Authorization Policies} Research on WSN security has addressed secure
routing \cite{senroute-ahnj03}, link layer security \cite{karlog-tinysec-2004}, cryptography
\cite{bertoni-2006} and key distribution \cite{camtepe-bulent-05}, and hardware issues
\cite{perrig-2004}. This previous work has established a strong low-level foundation for
security in WSNs. We believe that secure link-layer RPC is an appealing middleware solution in
WSNs because it allows programmatic specification of previously ad-hoc network-level behavior,
while imposing relatively small syntactic and efficiency overhead. SpartanRPC also allows
multiple symmetric keys to be used to protect multiple security domains within a network in a
simple and usable manner. Our security mechanism is discussed in more detail in
Sect.~\ref{section-security-extensions}.
