\section{Language Reference}

In this section we describe the specifics of the Spartan RPC language as an extension to nesC
version 1.2 \cite{nesC12}. Note that at this time only the control flow security features of
Spartan RPC are detailed here.

\pcnote{Eventually include the syntax of the Spartan RPC extensions}.

\subsection{Dynamic Wires}
\label{sec:dynamic-wires}

Each node has an ID that can be represented with a value of type \texttt{uint16\_t}. The way in
which nodes are assigned their IDs is outside the scope of our system. \pcnote{However we should
describe at least one plausible scenerio.} In addition each component on a node that can be
directly reached by a remote node is given an ID from the type \texttt{uint8\_t}. Notice that
this limits the number of remote-visible components on a node to 256. We do not expect this
limitation to be a problem in any forseeable application.

Component IDs are scoped by the ID of their node. Thus two different nodes can have components
with a local ID value of, for example, zero without conflict. In a given network, components
have unique IDs when this qualifcation is considered. This is represented by the following
structure.
\begin{verbatim}
typedef struct {
   uint16_t node_id;
   uint8_t  local_id;
} component_id;
\end{verbatim}

For convenience we define a type \texttt{component\_set} as a set of \texttt{component\_ids}.

\begin{verbatim}
typedef struct {
    component_id *ids;
    int           count;
} component_set;
\end{verbatim}

The field \texttt{ids} points at the base of an array of \texttt{component\_id} structures. The
field \texttt{count} is a count of the number of IDs in that array. The order of components in
the \texttt{ids} array is not significant. Also it is up to the application to ensure that a
particular component appears in the \texttt{ids} array only once. If a component appears more
than once the behavior is undefined.

A \textit{dynamic wire} is a ``fan out'' wiring where the programmer species a method for
obtaining the \texttt{component\_set} identifying the wire's multiple remote endpoints. This
method will be invoked at run time upon every dynamic traversal of that wire; the method may
return a different set upon every invocation. Formally, we specify this via the ComponentManager
interface.
\begin{verbatim}
interface ComponentManager
{ 
   command component_set elements();
}
\end{verbatim}

A component that provides this interface can be used to identify dynamic wire endpoints. This is
accomplished by first mentioning the component in the component-list of a configuration's
implementation and then using the square brackets around the component in the wiring syntax.

\begin{verbatim}
components N, C;
N.I -> [C].I;
\end{verbatim}

Here we connect N's used interface I with the I interfaces provided by all the components named
dynamically by C. Contrast this with a static fan-out configuration as follows

\begin{verbatim}
components N, C1, C2, C3;
N.I -> C1.I;
N.I -> C2.I;
N.I -> C3.I;
\end{verbatim}
    
In the dynamic case it is not feasible to list all the components involved since their number
may only be known at run time. Note the distinction in the two cases below:

\begin{verbatim}
components N, C;
N.I -> C.I;
N.I -> [C].I;
\end{verbatim}

In the first wiring N is connected to C's interface I. In the second wiring, N is connected to
the I interfaces of the components selected by C. The precise components involved is determined
each time a duty is posted over this wire and could be different from call to call.

There are several restrictions on this mechanism that are intended to simplfy the
implementation. First, only one level of indirection is allowed. For example the wiring
\texttt{N.I -> [[C]].I} is illegal. Supporting such a facility would increase the expressive
flexibility of the system by allowing dynamic wires to be specified by component manager
components on remote nodes. However, using this mechanism would be expensive in terms of radio
communication, and it does not seem worth the cost in energy and implementation complexity.

An additional restriction is that only duties (described in section \ref{sec:duties}) can be
posted over dynamic wires. There is no implementation support for call commands or signaling
events. This limitation allows the implementation to avoid the problems of providing synchronous
call behavior and of tracking the source of a dynamic wire so that replies can be returned.
Managing this in the general case is a significant burden.

Components that may be invoked over a dynamic wire must contain at least one remotable
interface.

\begin{verbatim}
module ServerC {
  provides remote interface I;
}
implementation {
  ...
}
\end{verbatim}

Note that generic components can not be remotable. If an instance of a generic component must be
made remotable it must be wrapped in an ordinary configuration

\begin{verbatim}
configuration Wrapper {
  provides remote interface I;
}
implementation {
  components new Queue(1, int);
  I = Queue.I;
}
\end{verbatim}

\subsection{Duties}
\label{sec:duties}

We further extend the nesC language to allow task-like entities, called duties, to appear in
interfaces and to be parameterized. Duties are declared in interfaces in a manner similar to
commands or events. They can take parameters but the types of those parameters are restricted to
``remotable types'' where a remotable type is defined inductively as follows:

\begin{itemize}
\item The integral types are remotable. Note that integral types include enumeration types.
\item Structures containing only remotable types or arrays of remotable types are remotable.
\end{itemize}

It is important to notice that neither pointer types nor array types (that are not nested inside
a structure) are remotable.

Interfaces that contain duties can only contain duties. A simple example to illustrate the idea
is the following:

\begin{verbatim}
interface Blink {
  duty void flash_led(int seconds);
}
\end{verbatim}

This interface describes a duty that when posted will cause one of the LEDs on the remote system
to flash for the indicated number of seconds. Note that duties must have a return type of void.
Their invocations may or may not occur, due to the inherent unreliability of wireless
communication, so there is little point returning even an error indication from them.

This duty can be posted in a manner similar to the way a task is posted except that arguments
are provided for the parameters in the obvious way.

\begin{verbatim}
module ClientC {
  uses interface Blink;
}
implementation {
  ...
  void f( )
  {
    ...
    post Blink.flash_led(2);
    ...
  }
  ...
}
\end{verbatim}

The post operation will cause the duty to be posted on all remote nodes connected to this module
via the dynamic wire specified in the appropriate configuration. Note that because of fan out
wiring, it is possible for a single post operation to actually post multiple duties in different
components. That is, each component to which the posting component is wired will post its own
implementation of the provided duty. However, unlike ordinary nesC tasks, these duties
potentially execute in parallel.

The component that provides this interface implements all the duties in the interface in the
usual away.

\begin{verbatim}
module ServerC {
  provides remote interface Blink;
}
implementation {
  ...
  duty void Blink.flash_led(int seconds)
  {
    // Make it work here.
  }
  ...
}
\end{verbatim}

\subsection{Access Control}
\label{sec:access-control}

Spartan RPC allows the programmer to control access to remotely exposed duties so that only
authorized clients can post them. By default a duty is publically accessible to all clients.
However, the programmer can restrict access to only those clients who can prove membership in a
specific role as defined by the $RT_0$ trust management system [citation needed].

\pcnote{Add a brief discussion of $RT_0$ here.}

The server programmer declares the role requirements in the module that provides the remote
interface. Note that security policy is defined by the providing module and not with the
interface itself. For example:

\begin{verbatim}
module ServerC {
  provides remote interface Blink requires "A.r";
}
implementation { ... }
\end{verbatim}

In this case the client must prove membership in role $A.r$ before any posting of duties in the
interface will be honored. In this context $A$ is an $RT_0$ entity (represented by a public key)
and $r$ is a role identifier.

Clients indicate which roles they will activate on each dynamic wire by decorating the dynamic
wire with a suitable ``auth'' indication. For example:

\begin{verbatim}
configuration AppC { }
implementation {
  components M, MyManagerC;
  
  auth "A.r" M.Blink -> [MyManagerC].Blink;
}
\end{verbatim}

Each time a duty is posted over this dynamic wire, the client will send $RT_0$ credentials to
the server that proves a particular public key $K$ is a member of role $A.r$. All subsequent
duty postings will be signed with key $K$. To do this the client node must wield the private key
that corresponds to $K$; this shows that the client is a legitmate member of the required role.

It is important to notice that the role specifications (the policy) is statically defined.
Spartan RPC does not provide any facility to dynamically compute role requirements. Also notice
that the required roles are specified by the server(s) involved. A different collection of
servers could provide the same interface but under different role requirements. However, every
server accessed via a particular dynamic wire should have the same role requirements or else
some servers will reject the corresponding duty postings.

