\section{Duties and Remotability}
\label{section-duties}

The wireless communication in a WSN is typically unreliable due to
frequent node failures and interference effects. Furthermore the data
transfer rate in a WSN based on the IEEE 802.15.4 link layer protocol
is much less than in other common networked environments. For example
the CC2420 radio used in our experiments, a typical WSN device,
supports a data rate of only 250\,Kbps
\cite{cc2420-datasheet}. Consequently we believe it is unrealistic for
RPC services in WSNs to be synchronous. Instead we believe the
semantics of nesC tasks are a more appropriate abstraction. They are
not quite right however, since RPC services, unlike nesC tasks, will
typically require arguments to be passed. Also a nesC task can only be
posted in the module where it is defined. In contrast an RPC service
invokes remotely defined functionality. We therefore define a new RPC
abstraction called a \emph{duty}.

\subsection{Syntax and Semantics}
\label{section-duties-syntax}

Duties are declared in interfaces and syntactically resemble command
declarations. Instead of using the reserved word \code{command} the new
reserved word \code{duty} is used. Duties are allowed to take
parameters, with restrictions as discussed below, but must return the
type \code{void}. For example the following interface describes an RPC
service for remotely controlling a collection of LEDs:

\begin{Verbatim}[fontsize=\small, commandchars=\\\[\]]
\centerline[interface LEDControl { duty void setLeds(uint8_t ctl); }]
\end{Verbatim}

Duties are defined in modules in a manner similar to the way tasks,
commands, or events are defined. The reserved word \code{duty} is again
used on the definition. Like commands and events the name of the duty is
qualified by the name of the interface in which it is declared.
Including a duty in an interface definition automatically implies that
the interface can be remotely invoked, or is \emph{remotable} in the
sense formalized in \autoref{section-remotable}. Any remotable interface
provided by a component must be specified as \code{remote} in the
component's list of provided interfaces. The first code sample in
\autoref{figure-duty-usage} shows an \code{LEDControllerC} component
that provides the \code{LEDControl} interface remotely-- i.e.~that
allows remote nodes to control LED status lights on a board. A more
extended example of duty implementation and usage is provided in
\autoref{section-example}.

A module on the client node that wishes to use a remote interface simply
posts the duty in the same manner as tasks are posted. The use of
\code{post} emphasizes the asynchronous nature of the invocation. An
example duty posting is illustrated in \autoref{figure-duty-usage}. The
standard component semantics of nesC provide a natural abstraction of
``where'' the RPC call goes, just as e.g.~a normal command invocation
will go through a component interface that is disconnected from its
implementation. Like a normal command invocation, configuration wirings
determine where duty control flows. However, in SpartanRPC duty
invocation control flows to a component residing on a different network
node. The invoking module must be connected to the remote modules by way
of a dynamic wire as described in \autoref{section-dynamic-wires}.

\begin{fpfig}[t]{Duty Implementation and Invocation Examples}{figure-duty-usage}
{
\begin{center}
%\begin{minipage}[t]{0in}
\vspace{0.5em}
\begin{Verbatim}[fontsize=\small]
         module LEDControllerC { provides remote interface LEDControl; }
         implementation {
           duty void LEDControl.setLeds(uint8_t ctl) { ... }
         } 
 
         module LoggerC { uses interface LEDControl; }
         implementation {
           void f() { ... post LEDControl.setLeds(42); }
         }
\end{Verbatim}
\vspace{0.1em}
%\end{minipage}
\end{center}
}
\end{fpfig}

When a duty is posted by a client it may run at some time in the future
on the server node. The client node continues at once without waiting
for the duty to start. Once posted the client has no direct way to
determine the status of the duty. Also, due to the unreliability of the
network a posted duty may not run at all. SpartanRPC does not attempt to
retransmit or even detect failed duty invocations; postings occur at
most once. Any error semantics for duty postings must be implemented by
the application developer.

\subsection{Remotable Interfaces}
\label{section-remotable}

We impose certain requirements on RPC service definitions for ease of
implementation. First, since WSN nodes do not share state we disallow
passing references to duties---such a reference would be meaningless on
the receiving node. Thus we define remotable types:
\begin{definition}A type is \emph{remotable} if and only if it satisfies
  the following inductive definition: The nesC built-in arithmetic
  types, including enumeration types, are remotable, and structures
  containing remotable types are remotable.
\end{definition}
Since a remotable interface describes RPC services, we require that they
specify duties taking only arguments of remotable type; also, remotable
interfaces can only contain duties, to ensure meaningful remote usage.
\begin{definition}
  An interface is \emph{remotable} if and only if it only provides
  duties whose argument types are remotable.
\end{definition}
