\section{Conclusion}

We have extended nesC with a light weight, link-layer, secure RPC facility, yielding a language
called SpartanRPC. SpartanRPC is a middleware technology supporting secure WSN applications
comprising multiple protection domains. It is ideal for settings in which multiple subnetworks
administered by distinct social entities cooperate to obtain a holistic behavior. A
language-level, capability-based authorization mechanism provides application programmers with
an easy and effective means for specifying and enforcing security policies.

Because of the long delays and unreliability inherent in radio communication, SpartanRPC treats
remote execution of RPC services as fundamentally asynchronous. Inspired by existing nesC
practice SpartanRPC provide task-like units of remote execution called \emph{duties}. In
addition SpartanRPC extends nesC configurations to allow components on different nodes to be
wired together in a dynamic manner, i.e.~remote wirings to RPC services can change during
program execution. This accommodates typical routing and programming patterns in WSN
applications.

We have implemented SpartanRPC in the Sprocket framework \cite{sprocket}, wherein RPC features
are transformed at compile into standard nesC code, and symmetric key cryptography and MACs
underlies the authorization mechanism. Empirical results suggest that SpartanRPC as implemented
in Sprocket is efficient and realistic for programming practice. We have illustrated the
facility of the language itself with an implementation of secure directed diffusion in a
heterogeneous trust environment.
