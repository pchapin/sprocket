%%%%%%%%%%%%%%%%%%%%%%%%%%%%%%%%%%%%%%%%%%%%%%%%%%%%%%%%%%%%%%%%%%%%%%%%%%%%
% FILE         : pchapin-notes.tex
% AUTHOR       : Peter C. Chapin <pchapin@cems.umv.edu>
% SUBJECT      : A few thoughts on RPC based security in WSNs.
%
%%%%%%%%%%%%%%%%%%%%%%%%%%%%%%%%%%%%%%%%%%%%%%%%%%%%%%%%%%%%%%%%%%%%%%%%%%%%

%+++++++++++++++++++++++++++++++++
% Preamble and global declarations
%+++++++++++++++++++++++++++++++++
\documentclass{article}
%\documentclass[12pt]{article}

% Common packages. Uncomment as needed.
%\usepackage{amsmath}
\usepackage{amssymb}
%\usepackage{amstext}
%\usepackage{amsthm}
%\usepackage{doublespace}
%\usepackage{fullpage}
%\usepackage[dvips]{graphics}
%\usepackage{listings}
%\usepackage{mathpartir}
%\usepackage{url}
%\usepackage{hyperref}

% The following are settings for the listings package.
%\lstset{language=Java,
%        basicstyle=\small,
%        stringstyle=\ttfamily,
%        commentstyle=\ttfamily,
%        xleftmargin=0.25in,
%        showstringspaces=false}

\setlength{\parindent}{0em}
\setlength{\parskip}{1.75ex plus0.5ex minus0.5ex}

% Some handy macros.
\newcommand{\todo}[1]{\textit{TODO: #1}}
\newcommand{\po}{\preccurlyeq}

%++++++++++++++++++++
% The document itself
%++++++++++++++++++++
\begin{document}

%-----------------------
% Title page information
%-----------------------
\title{RPC Based Security in Wireless Sensor Networks}
\author{Peter C. Chapin}
\date{February 19, 2008}
\maketitle

\section{Introduction}

This document is a somewhat random collection of thoughts and ideas I've had regarding the topic
of RPC security in wireless sensor networks (WSNs).

\section{Protection Domains, Components, and Guests}

Liu and Smith describe an infrastructure for securing access to ``cells'' using SDSI/SPKI
\cite{liu-smith-component-security-2002}. In their formulation a cell is a body of code that
provides and uses interfaces, much like a nesC component, and that can be connected to other
cells in an appropriate way. Their system applies access control at the boundary of each cell.
Doing something similar in the WSN case would suggest that each component in a nesC program
should have its own access policy. This is an interesting concept, but I believe it is much too
fine-grained for practical use in the WSN case.

Liu and Smith argue that cells are a natural boundary for security domains because they will
probably reflect the political boundaries of the organizations using a distributed system.
Another natural boundary, however, would be the nodes themselves. In this view, once access is
granted to a node, all components on that node are accessible. This level of security might be
too course grained for a large machine executing code for many security domains (for example a
multi-user server run by a web hosting company). However, there is zero chance of a WSN node
being used in this way.

My feeling is that an individual node is still too fine grained to define an appropriate
protection domain in the WSN case. An organization deploying a WSN will rarely deploy a single
node. Instead WSN protection domains should be entire networks (or perhaps subnetworks) of
nodes. In this respect how access control is managed between domains is, by its nature, a macro
property of the network. That is... from a security analysis point of view, we are only
interested in RPC calls from one network to another. RPC calls between individual nodes are
interesting only to the extent that they support inter-domain calls.

One important exception might be the base station. It is easy to imagine cases where a ``guest''
base station wishes to interact with a deployed WSN. The guest may wish to query information
known to the WSN (or visa-versa) and access control will have to be applied at that boundary.


\section{Key Management}

Due to limited computational resources on a WSN node, it is widely assumed that public key
cryptography is not an option for WSNs. This introduces some problems, however. Let $A$ and $B$
be two wireless sensor networks. Assume $A$ wishes to send an authenticated message to $B$.
Without public key cryptography some sort of symmetric key method must be used, such as a
message authentication code (MAC). To verify $A$'s MAC signature, $B$ will need a copy of $A$'s
signing key. As a side effect this allows $B$ to forge $A$'s signature.

In the usual scenario this isn't considered a problem because $A$ and $B$ are in the same
security domain and thus trust each other. The point of $A$'s signature is to protect against
packets being injected into the network from outsiders. However, if we now assume that $A$ and
$B$ represent networks in two different security domains, things change. Now $A$ will not want
$B$ to have a copy of her signing key since $A$ may not entirely trust $B$. Public key
cryptography solves this issue, of course, and that is exactly why it is important.

One possible way to deal with this is for $A$ and $B$ to use public key cryptography
``off-line'' to negotiate a symmetric session key $K_{ab}$ for use when authenticating messages
from $A$ to $B$. Note that for reasons that will be described shortly it will be necessary to
negotiate two keys: $K_{ab}$ and $K_{ba}$, one for authentication in each direction.

When I say off-line I mean that the managers of $A$'s security domain perform this negotiation
with the managers of $B$'s security domain over the ordinary Internet using traditional computer
systems. In practice this negotiation might occur automatically by the base stations of the two
networks.
\begin{enumerate}
\item $A$ wishes to call a procedure in $B$ but will need to authenticate the call. If $A$
  already has a $K_{ab}$, use that key and make the call normally.
\item If $A$ does not have a $K_{ab}$, $A$ contacts her base station requesting a session key
  for $B$.
\item If the base station (or one of the intermediate nodes in $A$'s network) has a $K_{ab}$, it
  is forwarded to the node that needs it.
\item Otherwise the base station engages in a protocol with $B$'s domain (for example, with
  $B$'s base station) to generate $K_{ab}$ and, if necessary, $K_{ba}$.
\item The two base stations propagate these keys to their respective networks.
\item Once $A$ gets $K_{ab}$, it makes the call normally.
\end{enumerate}

Note that the session keys should have a relatively long life span and should be shared among
all the nodes in a WSN. Thus the protocol above should only have to execute rarely. In fact, if
$A$ and $B$ know they are going to communicate ahead of time, the session key could be
negotiated and distributed before the networks are even deployed.

\subsection{Analysis}

Here is some informal analysis of the security of this scheme.

Assume $A$ is trying to communicate securely with two other WSNs $B$ and $C$. Since $K_{ab}$ and
$K_{ac}$ are different, it is not possible for $B$ to send a message to $C$ pretending to come
from $A$. That is, even though $B$ knows $K_{ab}$, that does not allow $B$ to find $K_{ac}$.
This assumes the session key negotiation protocol authenticates both parties (using public key
cryptography) so that $B$ can't execute that protocol pretending to be $C$ and thus convince $A$
to give him $K_{ac}$.

The nodes in $B$ are able to send forged messages (apparently coming from $A$) to each other
since all nodes in $B$ know $K_{ab}$. However, the nodes in $B$ are assumed to trust each other
so this is not a problem.

Now consider the case when $B$ sends a message signed with $K_{ab}$ back into network $A$. Can
$B$ confuse $A$ by sending it messages that appear to be coming from other nodes in $A$? $B$ can
not do this. The nodes in $A$ understand that messages signed with $K_{ab}$ are intended for
network $B$ and thus will ignore them. Messages from $B$ would be signed with a different key
$K_{ba}$ and network $A$ understands that such messages do not originate from other nodes in
$A$. This is why two session keys are necessary.

\bibliographystyle{plain}
\bibliography{references}

\end{document}
