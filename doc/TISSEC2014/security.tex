\section{Security Policy Specification and Program Logic}
\label{section-security-extensions}

In this section we discuss how to extend the language setting described
previously with security features. The goal is a language framework
where RPC services require authorization for use, and where
authorization policies support collaboration between multiple security
domains. To this end we adapt a distributed trust management system
\cite{chapin-skalka-wang-acmcs08} for policy specification. This system
secures WSN application programming by way of the SpartanRPC API.

\subsection{Security Policy Language}

Authorization in trust management systems is more expressive than in
traditional access control schemes such as access control lists or role
based access control (RBAC) \cite{Sandhu:RBACM}. In these simpler
models, access is based on identities of principals. But in the
distributed scenarios we are considering here, creating a single local
database of all potential requesters is untenable. Where there are
multiple domains of administrative control, no single authorizer can
have direct knowledge of all users of the system. Furthermore, in highly
dynamic and volatile environments, no single entity in-network can be
expected to keep pace with changes in an authoritative manner. Finally,
basing authorization purely on identity is not a sufficiently expressive
or flexible approach, since security in modern distributed systems
utilizes more sophisticated features (e.g.~delegation). These problems
are addressed by the use of trust management systems such as the \RT\
framework \cite{Li:DRBTMF}. We use the system $\RT_0$ in this
foundational presentation due to its simplicity, but other $\RT$
variants \cite{Li:DCDTM,Li:RRBTMF} could be adapted.

Like other trust management systems such as SPKI/SDSI \cite{RFC-2693},
\RT\ represents principals as public keys and does not attempt to
formalize the connection between a key and an individual. The \RT\
literature usually refers to principals as \newterm{entities}. \RT\
allows each entity to define \emph{roles} in a name space that is local
to that entity. An authorizer associates permissions with a particular
role; to access a resource a requester must prove membership in the
role. In this way \RT\ provides a form of role based access control.

To define a role, an entity issues credentials that specify the role's
membership. Some of these credentials may be a part of private policy,
others may be signed by the issuer and made publicly available as
certificates. The overall membership of a role is taken as the union of
the memberships specified by all known defining credentials.

Let $A, B, C, \ldots$ range over entities and let $r, s, t, \ldots$
range over role names. A role $r$ local to an entity $A$ is denoted by
$A.r$. \RT\ credentials are of the form $\cred{A.r}{f}$, where $f$ can
take on one of four forms to obtain one of four credential types:
\begin{longenum}

\item $\cred{A.r}{E}$. This form asserts that entity $E$ is a member of
  role $A.r$.

\item $\cred{A.r}{B.s}$. This form asserts that all members of role
  $B.s$ are members of role $A.r$. Credentials of this form can be used
  to delegate authority over the membership of a role to another entity.

\item $\cred{A.r}{B.s.t}$. This form asserts that for each member $E$ of
  $B.s$, all members of role $E.t$ are members of role $A.r$.
  Credentials of this form can be used to delegate authority over the
  membership of a role to all entities that have the attribute
  represented by $B.s$. The expression $B.s.t$ is called a \emph{linked
    role}.

\item $\cred{A.r}{q_1 \cap \cdots \cap q_n}$. Where the $q_i$ are
  qualified role names such as $B.s$. This form asserts that each entity
  that is a member of all roles $q_1,\ldots, q_n$ is also a member of
  role $A.r$. The expression $q_1 \cap \cdots \cap q_n$ is called an
  \emph{intersection role}. In our implementation only two constituent
  roles $q_1$ and $q_2$ are allowed in an intersection role. This does
  not limit expressivity since intermediate roles can be introduced as
  necessary to handle larger intersections.

\end{longenum}
For all credential forms $\cred{A.r}{f}$, the principal $A$ is called
the \emph{issuer} of the credential. An example credential set is
presented and discussed in \autoref{section-snowcloud}.

The formal semantics of \RT\ can be expressed in terms of \datalog\
\cite{Li:DRBTMF}. The translation of \RT\ credentials to \datalog\
requires only a single relation \textit{isMember} to assert when a
particular entity is a member of a particular role. A type (1)
credential, called a \newterm{membership credential}, is translated into
\datalog\ simply as a fact. For example the credential $\cred{A.r}{E}$
becomes the fact $\textit{isMember}(E, A, r)$. The other three
credential types are translated into \datalog\ rules. For example, the
type (3) credential $\cred{A.r}{B.s.t}$ becomes the following \datalog\
rule:\footnote{Logical variables are shown prefixed with `\textit{?}'}

\begin{displaymath}
\textit{isMember}(\textit{?x}, A, r) \leftarrow
  \textit{isMember}(\textit{?y}, B, s),
  \textit{isMember}(\textit{?x}, \textit{?y}, t).
\end{displaymath}

The meaning of an \RT\ credential $\semantics{C}$ is the \datalog\ fact
or rule to which it translates. Let $\creds$ be a set of \RT\
credentials split into two disjoint subsets $\creds = \creds_f \amalg
\creds_r$ where $\creds_f$ is the set of all membership credentials. The
meaning of $\creds$, which we denote as $\semantics{\creds}$, is the
minimum model of the \datalog\ program $\semantics{\creds_r}$ using
$\semantics{\creds_f}$ as input \cite{Abiteboul:FD}. The authorizer
associates an access permission with a particular role, say $A.g$, that
we call the \newterm{governing role}. Hence we formally define
authorization in a given credential environment $\creds$ as follows:

\begin{definition}
  Given a credential set $\creds$, entities $A$ and $E$, and role $g$,
  \emph{$E$ is authorized for $A.g$ in $\creds$}, denoted $\creds \vdash
  E \in A.g$, if and only if $\textit{isMember}(E, A, g)$ is in
  $\semantics{\creds}$.
\end{definition}

One appealing characteristic of the $RT_0$ trust management system is
monotonicity. Negative credentials that explicitly remove entities from
roles are not supported. Consequently if an authorizer has incomplete
information she might deny access that would otherwise be granted, but
she will never grant access that should have been denied. This property
is essential in a WSN context where the unreliability of wireless
communication together with the limited memory resources of sensor nodes
make it impossible to \emph{guarantee} complete information about all
roles.

Our system requires the requester to provide all necessary credentials
using some external means to obtain them. Methods for doing, for
example, distributed credential chain discovery have been described
\cite{Li:DCDTM} but we feel they would be prohibitively expensive in a
WSN context. The best approach for collecting a complete set of
credentials may be application specific, and thus we regard it as
outside the scope of this work.

The use of $RT_0$ means that principles cannot switch to roles of lesser
privilege, as is often desirable in accordance with the principle of
least privilege. However, variants of \RT\ are available that support
switching to less privileged roles \cite{Li:RRBTMF}, and that could be
used in SpartanRPC instead of $RT_0$. Also techniques for supporting
certificate revocation in an \RT-style trust management framework have
been explored \cite{lbi-fc01}.

\begin{comment}
  \paragraph{ECC Public Key Cryptography} Trust management systems such
  as \RT\ depend on public key cryptography. Distributed certificates
  are protected with digital signatures made using the issuer's private
  key. Furthermore public key cryptography is used to authenticate
  message senders to ensure that the entity requesting a service is the
  entity it claims to be. However, public key cryptography is
  computationally expensive and this presents a problem for limited
  devices such as wireless sensor network nodes.

  The on-node cryptographic computations required by our system are
  digital signature verification, session key negotiation using
  Diffie-Hellman key exchange, and message authentication code (MAC)
  computation using a session key. Of these three operations the first
  two involve complex public key cryptography. The MAC computation
  occurs much more frequently but is much cheaper since it uses hardware
  accelerated symmetric key cryptography. In fact, the motivation behind
  creating session keys is to avoid public key operations for every
  message.

  Although the RSA public key cryptosystem has been implemented on
  sensor nodes \cite{Gura04comparingelliptic,wang-2006}, the resource
  consumption required to do so is considerable. However, the
  feasibility of using elliptic curve public key cryptosystems (ECC) on
  such platforms has been repeatedly demonstrated
  \cite{1049776,Malan:2008:IPI:1387663.1387668,Liu-Peng-TinyECC-2008,Szczechowiak:2008:NTL:1786014.1786040}.
  Hardware implementations of ECC for resource limited devices have also
  been demonstrated \cite{kumar-2006,4604657}.

  ECC can achieve much higher security for a given number of key bits
  saving memory and network bandwidth relative to other public key
  cryptosystems. In our implementation, described in detail in
  \autoref{section-implementation}, we use 160 bit ECC keys, providing a
  security similar to 1024 bit RSA keys \cite{lenstra-verheul-2001}. We
  believe this is a reasonable level of security for the anticipated
  applications.
\end{comment}

\subsection{Program Logic}

Our authorization model can be viewed as a client-server interaction;
respective sides of the interaction protocol are summarized separately
as follows.

\subsubsection{RPC Server Side Logic}
\label{section-rpc-server-side}

RPC service providers establish policy by assigning governing roles
$A.g$ to remote interface implementations. Service providers also
possess a set of assumed credentials $\creds$ which establish an
authorization environment by providing initial server side authorization
policy. As we will describe in detail, the set $\creds$ may grow as
additional credentials are communicated to servers. Finally, in the
presence of security, client invocations of any RPC service are not
anonymous, but are performed on behalf of some entity $B$, which must be
a member of the governing role $A.g$ to use the protected service.

In summary, access to an RPC level is allowed if and only if the
property $\creds \vdash B \in A.g$ holds, where:
\begin{itemize}
  \item $B$ is the identity of the RPC client
  \item $A.g$ is the governing role of the RPC service
  \item $\creds$ are the credentials known to the RPC host server
\end{itemize}
RPC service programmers specify governing roles as part of module
definitions---specifically at remote interface \texttt{provides}
clauses. Hence, governing roles are associated with interface
\emph{implementations}, not interfaces themselves. This allows
application flexibility, in that the same interface can be implemented
with various authorization levels within the same network. Syntax is as
follows:

\begin{Verbatim}[fontsize=\small, commandchars=\\\{\}]
\centerline{provides remote interface \textit{I} requires \textit{A.g}}
\end{Verbatim}

Note the minor modification to previously introduced syntax for remote
module definitions via the \texttt{requires} keyword.

\subsubsection{RPC Client Side Logic}
\label{section-rpc-client-side}

In order to use a secure remote module, RPC clients wire to it as for
unsecured modules (see \autoref{section-wiringsyntax}), but with two
additional capabilities: (1) the client specifies under what \RT\ entity
the invocations will be performed, and (2) the client may also specify
credentials in their possession which are to be activated for use in the
invocation. Syntax is as follows:

\begin{Verbatim}[fontsize=\small, commandchars=\\\{\}, codes={\catcode`*=3\catcode`!=8}]
\centerline{enable "*C!1, \ldots, C!n*" as "*B*" for C.I -> [M].J}
\end{Verbatim}

For any invocation made through this wiring the credentials $C_1,
\ldots, C_n$ will be remotely added to the RPC server's database for the
authorization decision, via a process detailed in
\autoref{section-implementation}. Note that these credentials \emph{need
  not establish authorization entirely by themselves}, rather they will
be \emph{added} to the server's existing credentials, all of which will
be used in the authorization decision. A special form of the
\code{enable} clause using \code{"*"} for the list of credentials is
also supported. This form indicates that all credentials known to the
client should be communicated to the server.

Each node is deployed with a collection of ECC key pairs, one for each
entity the node represents. When an invocation is made the entity $B$
mentioned in the \code{as} clause of the dynamic wire is used in the
request. The \code{as} clause is optional; if it is omitted a
distinguished \newterm{default entity} is used for the invocation.

% The reader will also note the syntax \texttt{assuming "$D_1, \ldots,
% D_k$"} above, which denotes the client's assumptions about credentials
% already known to the RPC server(s). Although this language does not
% have run-time relevance per se, its purpose is to allow static
% checking of enabled credentials on the client side. That is, the
% client possesses static knowledge of the governing role $A.r$ required
% to use modules provided as $[M].J$. \cnote{There is a problem here,
% how will the client know this? Note that this is a different matter
% than knowing about interfaces, which are ``constant''-- we have
% already allowed that different modules may implement interfaces at
% different security levels, so how do we ``keep track'' of them? It
% would be nice to resolve this, since static checking of this form is
% an appealing feature...} The system will statically verify the
% property $$C_1, \ldots, C_n, D_1, \ldots, D_k \vdash B \in A.r$$ and
% report an error in case the specified credentials are insufficient to
% establish authorization. Note that the client need not actually
% possess credentials $D_1, \ldots, D_K$ for this check to succeed;
% also, it may be the case that these credentials are not in fact known
% by the server, so that a successful check does not guarantee runtime
% authorization.

\begin{comment}
  \subsubsection{Example}
  \label{section-security-example}

  Suppose that an existing network deployment \code{NetA} is imaged with
  a component \code{SamplingRateC} which provides a means to control
  sampling rates through an interface \code{SamplingRate}. Further,
  since sampling rate modification is a sensitive operation, the network
  administrators require \code{NetA.control} authorization to use this
  component:

  \begin{lrbox}{\savebigbox}
  \begin{minipage}{5.0in}
  \vspace{0.6em}
  \begin{Verbatim}[fontsize=\small]
  module SamplingRateC {
      provides remote interface SamplingRate requires "NetA.control";
  }
  \end{Verbatim}
  \vspace{0.3em}
  \end{minipage}
  \end{lrbox}
  \centerline{\usebox{\savebigbox}}

  \noindent Any node supporting this component will transparently
  receive \RT\ credentials from neighboring nodes and attempt to use
  those credentials to establish that each client entity is a member of
  the \code{NetA.control} role in the formal sense described above.

  Suppose also that nodes in \code{NetA} are deployed with the credential

  \begin{displaymath}
  \code{NetA.control} \leftarrow \code{WSNAdmin.control}
  \end{displaymath}

  Here the role \code{WSNAdmin.control} is administered by some
  overarching network authority. However this authority need not be
  physically ``present'' in the network during operation. Instead the
  credential above represents \code{NetA}'s access control policy: any
  entity blessed by \code{WSNAdmin} as a controller can control
  \code{NetA}.

  Suppose further that another subnet, called \code{NetB}, wishes to
  modify the sampling rate of \code{NetA}. A node in \code{NetB} might
  be imaged with the following credentials, among possibly others:

  \begin{eqnarray}
  \code{WSNAdmin.control} & \leftarrow & \code{NetB.control} \\
  \code{NetB.control}     & \leftarrow & \code{NetB}
  \end{eqnarray}

  Note that credential (1) is issued by the \code{WSNAdmin} authority,
  while credential (2) is issued by \code{NetB}. Critically, direct
  communication with \code{NetA} authorities to obtain these credentials
  is unnecessary.

  In order to invoke this service the wiring as shown in
  \autoref{figure-secure-wire} could be made on the client side. Note
  the activation of the necessary credentials, as well as the
  specification of client identity as \code{NetB}.

  \begin{figure}[!t]
  \begin{textbox}{3.0in}
  \begin{Verbatim}[fontsize=\small]
  enable
    "WSNAdmin.control <- NetB.control, 
      NetB.control    <- NetB" as "NetB"
  for 
    ClientC.SamplingRate -> [RemoteSelectorC];
  \end{Verbatim}
  \end{textbox}
  \caption{Security Enabled Dynamic Wire}
  \label{figure-secure-wire}
  \end{figure}

\end{comment}
