\section{Usage}

This section describes how to use Sprocket to build Spartan RPC enabled programs. It describes
command line options, error messages, and issues regarding the interaction with nesC and the
TinyOS build system.

\subsection{Compilation Model}

Sprocket is a preprocessor. It accepts an extended dialect of nesC and outputs pure nesC that is
then subsequently translated with the regular nesC compiler. Sprocket does not (currently)
invoke the back-end nesC compiler itself. You must do that manually or, more likely, with the
help of a suitable makefile.

Sprocket analyzes the entire program and makes transformations on wiring configurations that
require knowledge of all parts of the program. Sprocket does not normally process individual
files separately. Thus it is necessary for all the files composing a program to be in locations
that are known to Sprocket. Currently all input files must reside in a single folder (but note
that "library" components are given special handling). Sprocket must be run in this folder and
it transforms the program into a pure nesC program that it writes to a different folder. The
output folder need not exist when Sprocket starts, but if it does exist it is erased first. This
is done to ensure that the final output folder only contains files from the current run and no
files from previous runs.

Sprocket first runs the standard C preprocessor over all the input files and places the
resulting preprocessed files into a temporary folder. This allows preprocessor macros,
\code{\#include} directives, and conditional compilation directives to all be honored before
Sprocket begins its analysis. Sprocket then rewrites the program from this temporary folder to
the final output folder. As with the output folder, if the temporary folder does not exist it is
automatically created. If it already exists, it is erased first to avoid confusing Sprocket with
leftover files from previous runs.

It should be noted that fully preprocessing the input before compiling any files contradicts the
compilation model of nesC. Unlike Sprocket, the nesC compiler interleaves preprocessing and
compilation as it processes the components and interfaces that comprise a program. The effect of
this is that Sprocket currently fails to compile certain programs that should be legal under
nesC's compilation model. We hope to correct this in the future by integrating the preprocessing
step into Sprocket in a manner similar to the way it is integrated into the normal nesC
compiler. Right now this can be worked around using a suitable programming convention: macros
and types needed by a component or interface should be included directly in the file defining
that component or interface.

By default Sprocket uses an output folder of \filename{Sprocket-Out} and a temporary folder of
\filename{Sprocket-Tmp}, both in the current folder. However, these defaults can be overridden
using the configuration file or command line options. See the sections below for more
information.

\subsection{Configuration File}

Sprocket reads a configuration file where its settings can be defined. By default this file is
named \filename{.sprocket} in the invoking user's home directory. However, this default name and
location can be overridden using the \texttt{-config} command line option.

The configuration file consists of a collection of \texttt{NAME="VALUE"} pairs. Each pair is
defined on a single line. Leading spaces, trailing spaces, and spaces around the \texttt{=} are
ignored. Names can consist of letters, digits, or underscores. Values can contain any character,
but \emph{must} be quoted in all cases. Blank lines are ignored, and comments starting with a
\texttt{\#} character and running to the end of the line are ignored.

The following is a list of permitted configuration settings.

\begin{description}

\item[DebugMode] The value of this setting can be either ``true'' or ``false.'' When true
  Sprocket enters a special debugging mode that is only of interest to Sprocket developers. In
  this mode it only processes a single file as specified by the SourceFile configuration
  setting. The default value of this setting is ``false.''

\item[IncludePaths] The value of this setting is a colon separated list of folders used to
  resolve \code{\#include} directives during preprocessing. This list is passed to the
  preprocessor program.

\item[InputFolder] The value of this setting specifies the path to the input folder. This folder
  contains a complete Spartan RPC program for transformation. The default value of this setting
  is the current folder.

\item[NodeEntities] A comma separated list of entity names that the node can use. The first name
  in the list is the default entity that is used on a dynamic wire without an \texttt{as}
  clause.

\item[OutputFolder] The value of this setting specifies the path to the output folder. This
  folder will contain a pure nesC program that is the transformed result of the Spartan-RPC
  program in the input folder. The output folder is created if it does not exist. It is erased
  if it does exist. The default value of this setting is \filename{Sprocket-Out}.

\item[Preprocessor] The value of this setting specifies the C preprocessor to use. The name can
  be given with a path, either absolute or relative, or it can be found using the PATH
  environment variable. However, the value of this setting is restricted to contain only the
  name of a program. Do not include any additional options for the program here. The default
  value of this setting is \filename{cpp}. We recommend that any preprocessor used be compatible
  with \filename{cpp} in the sense of accepting the same command line options and producing the
  same output with respect to \#line directives.

\item[ShowSettings] The value of this setting can be either ``true'' or ``false.'' When true
  Sprocket just displays its configuration settings and halts. It does not process any files.
  This setting is useful for debugging Sprocket's configuration; it allows you to view the
  overall configuration being used without committing yourself to any operations.

\item[SourceFile] The value of this setting is the name of a single file for Sprocket to
  process. It is only allowed in debug mode (and in fact, must be used in debug mode). Normally
  Sprocket processes all the files in a program but during development it is sometimes useful to
  exercise the program on a single input file.

\item[TemplateFolder] The value of this setting specifies the path to Sprocket's library of run
  time component templates. Sprocket uses these templates to generate the specialized run time
  components needed to support the RPC abstraction. These templates are part of Sprocket's
  distribution; they are not intended to be edited by end users. There is no default value for
  this setting.

\item[TemporaryFolder] The value of this setting specifies the path to the temporary folder.
  This folder will contain the preprocessed version of the Spartan RPC program in the input
  folder. The temporary folder is created if it does not exist. It is erased if it does exist.
  The default value of this setting is \filename{Sprocket-Tmp}.

\end{description}

\subsection{Command Line Syntax}

Sprocket is a Scala program and thus must be executed in a Java virtual machine. The general
syntax is
\begin{verbatim}
java -jar Sprocket.jar [options]
\end{verbatim}

Without any options or non-default configuration settings, Sprocket transforms the Spartan-RPC
program in the current folder into a pure nesC program in the subfolder \filename{Sprocket-Out}
using a temporary subfolder \filename{Sprocket-Tmp}. If Sprocket encounters an error during the
run, the program left in these folders may be incomplete. Note that the temporary folder is not
deleted at the end of the run, allowing you to inspect the preprocessed version of your program.
This is useful for debugging both your code and Sprocket itself.

Options, if present, are prefixed with a dash character and normally have names consisting of a
single letter. Some options may have parameters. Option parameters are separated from an
option's name by an equals sign with no surrounding spaces. Each option is separated from the
others by at least one space. The order in which options appear is not significant. Except as
mentioned below, if the same option occurs more than once, the last occurrence is the one that
is used. For example
\begin{verbatim}
java -jar Sprocket.jar \
  -i=Sprocket-In -o=Result -n -i=Input
\end{verbatim}
The command above specifies an input folder of \filename{Input}, an output folder of
\filename{Result} (both relative to the current folder), and the \texttt{-n} option.

Note that without any options or explict configuration settings, Sprocket behaves as if it was
given the command line
\begin{verbatim}
java -jar Sprocket.jar \
  -i=. -t=Sprocket-Tmp -o=Sprocket-Out -p=cpp
\end{verbatim}

Before processing the command line, Sprocket can optionally read a configuration file (see the
section above). The configuration file can be used to set many of Sprocket's options. If a
command line option conflicts with a configuration file setting, the command line option
overrides the configuation file.

The following options are supported.

\begin{description}

\item[-config] Takes a parameter that specifies the path (relative or absolute) to the
configuration file. If no configuration file is specified, the file \filename{.sprocket} in the
user's home directory is used by default. The configuration file does not need to exist. If it
does not, no error message is produced. Note that, if present, this option is processed first
then the configuration file is read and finally the other command line options are processed.

\item[-d] Activates Sprocket's \textit{debug mode}. The use of this mode is described in more
detail in the following section.

\item[-f] Provides the name of a single input file. This option is only used when Sprocket is in
debug mode. See the following section for more information.

\item[-i] Provides a value for the InputFolder configuration setting.

\item[-I] Provides a value for the IncludePaths configuration setting. Note, however, that
unlike the configuration setting multiple occurrences of \texttt{-I} on the command line are
cumulative. Folders mentioned by later occurrences are appended to the list created by earlier
occurrences. In addition, folders mentioned with \texttt{-I} are appended to the list specified
by the \texttt{IncludePaths} configuration setting (instead of overridding that setting).

\item[-m] Provides a value for the TemplateFolder configuration setting.

\item[-n] Provides a value for the NodeEntities configuration setting.

\item[-o] Provides a value for the OutputFolder configuration setting.

\item[-p] Provides a value for the Preprocessor configuration setting.

\item[-s] Displays the various configuration settings that would be used by Sprocket. No further
processing is done. In particular, no files are erased and no files are read. This option can be
useful for debugging to get a better understanding of what Sprocket is trying to do in a
particular context. It shows the result of combining the configuration file settings with the
command line options.

\item[-t] Provides a value for the TemporaryFolder configuration setting.

\end{description}

\subsection{Debug Mode}

Sprocket provides a special debug mode of operation that can be used by Sprocket developers to
debug issues with Sprocket itself. This mode is not intended for use by end users. In debug
mode, Sprocket accepts the name of a single file on the command line by way of the \texttt{-f}
option. It preprocesses this one file to the temporary folder and then analyzes this file in the
usual way (although certain aspects of the analysis may be incomplete since the entire program
is not processed in this mode). After analyzing the file, Sprocket outputs certain internal
information about the file and about Sprocket's state.

Note that in debug mode, Sprocket does not use or erase the output folder.

