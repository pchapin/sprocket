\section{Duties and Remotability}
\label{section-duties}

Because of the slow, unreliable nature of wireless communications we believe it is unrealistic
for RPC services in WSNs to be synchronous. Instead we believe that the semantics of tasks are
closer to being a correct abstraction. They are not quite right however, as RPC services will
typically require arguments to be passed, and while the poster of a task defines it, an RPC
service invokes remotely defined functionality. We therefore define a new RPC abstraction called
a \emph{duty}.

\subsection{Syntax and Semantics}
\label{section-duties-syntax}

Duties are declared in interfaces and syntactically resemble command declarations. Instead of
using the reserved word \code{command} the new reserved word \code{duty} is used. Duties are
allowed to take parameters (with restrictions as discussed below) but must return the type
\code{void}. For example the following interface describes an RPC service for remotely flashing
a mote LED:
\begin{Verbatim}
interface LEDControl {
    duty void setLeds(uint8_t ctl);
}
\end{Verbatim}

Duties are defined in modules in a manner similar to the way tasks, commands, or events are
defined. The reserved word \code{duty} is again used on the definition. Like commands and events
the name of the duty is qualified by the name of the interface in which it is declared.
Including a duty in an interface definition automatically implies that the interface can be
remotely invoked, or is \emph{remotable} in the sense formalized in
Sect.~\ref{section-remotable}. Any remotable interface provided by a component must be specified
as \code{remote} in its provides specification, for example:
\begin{Verbatim}
module LEDControllerC {
    provides remote interface LEDControl;
}
implementation {
    duty void LEDControl.setLeds(uint8_t ctl)
    { ... }
}
\end{Verbatim}

A module on the client node that wishes to use a remotable interface simply posts the duty in
the same manner as tasks are posted. The use of \code{post} emphasizes the asynchronous nature
of the invocation.
\begin{Verbatim}
module LoggerC {
    uses interface LEDControl;
}
implementation {
    ...
    post LEDControl.setLeds(42);
}
\end{Verbatim}

Note that the standard component semantics of nesC provide here a natural abstraction of
``where'' the RPC call goes, just as e.g.~a normal command invocation will go through a
component interface that is disconnected from its implementation. Like a normal command
invocation, configuration wirings determine where duty control flows. However, in SpartanRPC
duty invocation control may flow to a component residing on a different network node. The
invoking module must be connected to the remote modules by way of a dynamic wire as described in
Sect.~\ref{section-dynamic-wires}.

When a duty is posted by a client node it may run at some time in the future on the server node.
The client node continues at once without waiting for the duty to start, i.e.~duty postings are
asynchronous in the same manner that tasks are. Once posted the client has no direct way to
determine the status of the duty. Also, due to the unreliability of the network a posted duty
may not run at all.

It is possible for a duty to be posted multiple times by a client or by multiple clients.
Because duties are implemented as nesC tasks as discussed in Sect.~\ref{section-implementation},
any posts of a particular duty received by a node while a previous post of that duty is pending
are lost. However, this does not introduce any new problems because duty execution is not
guaranteed in any case.

\subsection{Remotable Interfaces}
\label{section-remotable}

We impose certain requirements on RPC service definitions for ease of implementation. First,
since WSN nodes do not share state we disallow passing references to duties---such a reference
would be meaningless on the receiving node. Thus we define remotable types:
\begin{definition}A type is \emph{remotable} iff it satisfies the following inductive
  definition: The nesC built-in arithmetic types, including enumeration types, are remotable,
  and arrays of remotable types and structures containing remotable types are remotable.
\end{definition}
Since a remotable interface describes RPC services, we require that they specify duties taking
only arguments of remotable type; also, remotable interfaces can only contain duties, to ensure
meaningful remote usage.
\begin{definition}
  An interface is \emph{remotable} iff it only provides duties whose argument types are
  remotable.
\end{definition}
