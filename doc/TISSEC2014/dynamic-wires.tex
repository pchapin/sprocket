\section{Dynamic Wires}
\label{section-dynamic-wires}

In an ordinary nesC program the ``wiring'' between components as defined
by configurations is entirely static. The nesC compiler arranges for all
connections, and at run time the code invoked by each called command or
signaled event is predetermined.

In a remote procedure call system for wireless networks this static
arrangement is insufficient. A node can not, in general, know its
neighbors at compilation time but rather must discover this information
after deployment. In addition, the volatility of wireless links and of
the nodes themselves means that a given node's set of neighbors will
change over time. In this section we discuss the facility in SpartanRPC
to allow \emph{dynamic wirings} for specifying control flow from duty
invocation to duty implementation.

\subsection{Component IDs, Component Managers}
\label{section-componentmanager}

We begin by discussing how remote components are identified for wiring.
In order to uniquely identify components on the network, remotable
components are specified via a two-element structure called a
\code{component\_id} defined on the left side of
\autoref{figure-componentmanager}. The \code{node\_id} member is the
same node ID used by TinyOS and is set when the node is programmed
during deployment. The local ID member is an arbitrary value defined by
the programmer of the server node. Only components that are visible
remotely need to have ID values assigned, however, the ID values must be
unique \emph{on the node}. The \code{component\_set} structure defined
on the right side of \autoref{figure-componentmanager} wraps an
arbitrary array of \code{component\_id} values.
 
\begin{figure}[!t]
\begin{textbox}{4in}
\begin{minipage}[t]{1.75in}
\begin{Verbatim}[fontsize=\small]
    typedef struct {
        uint16_t node_id;
        uint8_t  local_id;
    } component_id;
\end{Verbatim}
\end{minipage}
\hfill
\begin{minipage}[t]{1.75in}
\begin{Verbatim}[fontsize=\small]
typedef struct {
    int count;
    component_id *ids;
} component_set;
\end{Verbatim}
\end{minipage}
\\
\centering
\begin{minipage}[t]{5.8in}
\vspace{1.5em}
\begin{Verbatim}[fontsize=\small]
interface ComponentManager { command component_set elements(); }
\end{Verbatim}
\end{minipage}
\end{textbox}
\caption{Component Manager Interface and Type Definitions}
\label{figure-componentmanager}
\end{figure}

A \emph{component manager} is a component that provides the
\code{ComponentManager} interface defined at the bottom of
\autoref{figure-componentmanager}. It dynamically specifies a set of
component IDs that ultimately serve as dynamic wiring endpoints. An
example component manager is discussed in detail
\autoref{section-example}.

As a simple example, consider the component manager
\code{RemoteSelectorC} as shown in
\autoref{figure-example-componentmanager}. This component manager always
returns a component set containing a single component. The special
SpartanRPC broadcast node ID is used (\code{0xFFFF}) indicating that all
neighbors should be the target of the dynamic wire. The component ID on
the neighbors is specified as $1$ in this example. In a more complex
example the component manager would compute the component set each time
the dynamic wire is used, filling in an array of component IDs based on
information gathered earlier.

\begin{figure}[!t]
\begin{textbox}{4.0in}
\begin{Verbatim}[fontsize=\small]
module RemoteSelectorC { provides interface ComponentManager; }
implementation {
    component_id  broadcast  = { 0xFFFF, 1 };
    component_set broadcast_set = { 1, &broadcast };

    command component_set ComponentManager.elements() {
        return broadcast_set;
    }
}
\end{Verbatim}
\end{textbox}
\caption{Example Component Manager}
\label{figure-example-componentmanager}
\end{figure}

\subsection{Syntax and Semantics}
\label{section-wiringsyntax}

In SpartanRPC we extend the syntax and semantics of nesC to allow the
target of a connection to be dynamically specified by a component
manager. The syntax of wirings, or connections, is extended as follows:

\begin{lrbox}{\savebigbox}
\begin{minipage}{4.3in}
\vspace{0.6em}
\begin{Verbatim}[fontsize=\small]
        connection ::= endpoint '->' dynamic_endpoint
  dynamic_endpoint ::= '[' IDENTIFIER ']' ('.' IDENTIFIER)?
\end{Verbatim}
\vspace{0.3em}
\end{minipage}
\end{lrbox}
\centerline{\usebox{\savebigbox}}

Given a dynamic wiring of the form \code{C.I -> [M].I}, we informally
summarize its semantics as follows. First, we statically require that
\code{M} be a component providing the \code{ComponentManager} interface,
and that \code{I} be a remotable interface. At run time, if control
flows across this wire via posting of some duty \code{I.d} within
component \code{C}, the command \code{elements} in \code{M} is called to
obtain a set of component IDs. The duties \code{I.d} provided by the
specified remote components will then be posted on their respective
nodes via an underlying link layer communication, the details of which
are hidden from the programmer. Thus, duties can only be posted on
neighbors. Note that since this call to \code{elements} may return more
than one component ID, this is a sort of fan-out wiring.

For example, the programmer could wire the \code{LoggerC} component
mentioned in \autoref{figure-duty-usage} to LED controller components on
a dynamically changing subset of neighbors using a configuration such
as:
\begin{Verbatim}[fontsize=\small, commandchars=\\\{\}]
\centerline{LoggerC.LEDControl -> [RemoteSelectorC];}
\end{Verbatim}

The server's configuration does not need to wire anything to the remote
interface explicitly.

\subsection{Callbacks and First-Class IDs}

We assume that the component IDs for well known services will be agreed
upon ahead of time by a social process outside of our system. By
broadcasting to a well known component ID, a node can use services on
neighboring nodes without knowing their node IDs.

If a node expects a reply from a service that it invokes, the invoking
node must set up a component with a suitable remote interface to receive
the service's result. In SpartanRPC remote invocations can only transmit
information in one direction. Bidirectional data flow requires separate
dynamic wires. This design provides a natural ``split-phase'' semantic
in which the invoker of a service can continue executing while waiting
for the result of that service. For example, a service might require the
client to provide the node ID and component ID of the component that
will receive the service result as arguments to the service invocation.
The server could store those values for use by a server-side component
manager. It is permitted for a component to be its own component manager
making it easy for a service to return a result by posting the
appropriate duty.

% For example, assume that the LED controller on the server returns the
% old state of the LEDs whenever the LED value is changed. The server
% configuration would include an appropriate dynamic wire as follows
%
% \begin{Verbatim}[commandchars=\\\{\}, fontsize=\small]
% \centerline{LEDControllerC.LEDResult -> [LEDControllerC];}
% \end{Verbatim}
%
% The client must provide the LEDResult interface remotely to receive
% this result. In this example the \code{LEDControllerC} component is
% its own component manager. This makes it easy for the \code{elements}
% command to access global data that was recorded inside
% \code{LEDControllerC} when the service it provides was previously
% invoked. This is a common SpartanRPC idiom.
