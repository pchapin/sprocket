
%
% NOTES ON RAW DATA COLLECTED
% ===========================
%
% Without low power listening the client TX time varied from a minimum of about 3 mS to a
% maximum of about 10 mS. (This was with, for example, the Baseline client). I assume this
% variation is due to variations in TinyOS latency. It completely swamps the packet overhead due
% to the MAC.
%
% Current sensing resistor = 14.9 $\Omega$
% Baseline and Duties client were identical in the measurements below.
%     TX pulse width 15-16 ms
%     Vsense = 276 mV
%     Isense = 18.5 mA
%     Vmote  = 3.00 - 0.276 = 2.724
%     Pmote  = 50.4 mW
%     Compute burst: 28.0 mV => 1.88 mA => (3.00 - 0.028)*1.88mA = 5.59 mW
%                    Duration = 3 ms => 16.8 uJ

% Security client has an identical TX pulse profile.
%     Compute burst: 28.0 mV => 1.88 mA => (3.00 - 0.028)*1.88mA = 5.59 mW
%                    Duration = 4 ms => 22.4 uJ

\section{Empirical Results}
\label{section-empirical-results}

The sensor nodes that are the target of our system are highly constrained devices with limited
memory, CPU resources, and electrical power. Conserving the resources of the platform is a
matter of high importance. Our system consumes additional resources because of the overhead
needed to manage remote procedure calls and cryptographic operations.

\subsection{Test Programs}

To explore the performance of our system we conducted several experiments. Our test devices were
two Tmote Sky wireless sensor nodes \cite{tmotesky-datasheet}. These devices are based on the
Texas Instruments MSP430F1611 microcontroller \cite{msp430-datasheet} with 48 KiB of flash ROM,
10 KiB of static RAM, and running at a clock frequency of 8 MHz. For wireless communication each
node uses a 2.4 GHz IEEE 802.15.4 Chipcon CC2420 \cite{cc2420-datasheet} low power transceiver.
The system software we used was TinyOS version 2.1.0 \cite{tinyos}.

The client node executed a program that periodically invoked a service on the server node,
passing that service an eight bit value. The server used the least significant three bits of
that value to control the LEDs on the server mote. Several pairs of test programs were written
that all performed the same essential function but in progressively more abstract ways. The
``baseline'' programs did not use any of our extensions. All radio handling was done explicitly
and no cryptography was used. The ``duties'' programs used dynamic wires and duties for RPC
support. The ``security'' programs added the authorization support and exercised the full
functionality of our system.

In a realistic context the average energy consumption of the radio can often be reduced by using
\emph{low power listening} \cite{moss-tep105}. In this mode the receiver runs with the radio off
for significant periods of time, waking the radio up periodically to see if any remote devices
are transmitting on the channel. In addition the transmitter broadcasts for a certain amount of
extra time to ensure that its transmission will properly overlap with at least one receiver
cycle. We used this mode in our test programs. As a result the extra transmission and reception
energy required for each data packet completely overshadowed overhead due to the SpartanRPC
extensions.

Since our system hides the radio handling from the application, the use of low power listening
becomes an issue for Sprocket to handle. Our implementation assumes that low power listening
will be used in all cases and writes the stubs and skeletons accordingly. Sprocket currently
uses a radio sleep interval of 10 ms with a 50\% duty cycle. When not actively transmitting
Sprocket turns the transmitter radio off.

\subsection{Memory Consumption}

Table~\ref{tbl:memory-consumption} shows the overall memory consumption, as reported by the nesC
compiler, of each test program.

\begin{table}[!t]
\newcommand\T{\rule{0pt}{2.1ex}}
\centering
\caption{Memory consumption of test programs.}
\label{tbl:memory-consumption}
\begin{tabular}{|l|rr|rr|} \hline
                   & \multicolumn{2}{|c|}{ROM \T} & \multicolumn{2}{|c|}{RAM} \\ \hline
                   & Bytes \T & \% & Bytes & \%  \\ \hline \hline
Baseline Client \T & 13096 & --- & 378  & --- \\ \hline
Baseline Server \T & 12576 & --- & 306  & --- \\ \hline \hline
Duties Client   \T & 13568 & 3.6 & 398  & 5.3 \\ \hline
Duties Server   \T & 12624 & 0.4 & 308  & 0.6 \\ \hline \hline
Security Client \T & 22662 & 73  & 608  & 61  \\ \hline
Security Server \T & 21978 & 75  & 534  & 74  \\ \hline
\end{tabular}
\end{table}

The dramatic increase in memory consumption of the security enabled programs is a consequence of
the pure software AES implementation used for the cryptographic operations
\cite{erdelsky-rijndael-2002}. The RAM increase is largely due to the buffer space required for
encryption and may be subject to further optimization. Most of the extra code and data space can
be reused for each security enabled remote invocation. Support for security on additional
dynamic wires or remote interfaces requires only about the same amount of overhead as is
required for the plain duty case.

\subsection{Energy Consumption}

We also evaluated the energy consumption of the three client programs. This was done by placing
a 14.9 $\Omega$ current sensing resistor in series with the 3.0 V power supply and observing the
power supply current waveform on an oscilloscope.

In our experiments the transmitter pulse width varied between 15 and 16 ms, and was the same for
all three programs. The energy consumed during transmission was 780 $\mu$J. Just before the
transmitter pulse started, a small increase in power supply current was observed lasting for 3
ms (in the baseline and duties-only case) to 4 ms (in the security enabled case). We assume this
increase is due to activities on the node that are done just prior to enabling the radio for
transmission. This \newterm{compute burst} consumed 17 $\mu$J of energy in the first two
programs and 22 $\mu$J of energy in the security enabled case. Presumably the additional energy
used in the security case is at least partially due to the cryptographic computations.

Despite our term ``compute burst,'' it is likely that some of this energy was actually being
used to power up various supporting components related to the upcoming transmitter pulse. For
example the CC2420 radio requires that its voltage regulator and oscillator be turned on and
allowed to stabilize for a time before signals can actually be transmitted
\cite{cc2420-datasheet}. For example, Even when fully active the microcontroller draws a maximum
current of only about 500 $\mu$A \cite{msp430-datasheet}. In our environment a 500 $\mu$A
current burst of 4 ms corresponds to just 6 $\mu$J of energy which is clearly a minority of the
observed energy.
