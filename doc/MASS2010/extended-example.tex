\section{Example}
\label{section-example}

To illustrate the usefulness of our design we implemented a skeleton program that uses directed
diffusion to gather temperature events in a heterogeneous network. The directed diffusion
algorithm requires that nodes communicate with a dynamically changing subset of neighbors. The
dynamic wire mechanism of our system makes it straightforward for a component manager to compute
the subset of neighbors currently needed in each communication. \emph{Who} receives a
communication is computed independently from \emph{what} is communicated.

In addition, the algorithm requires the use of two distinct communication pathways. Nodes
interested in receiving data communicate their interest forward over the network toward sensors.
Nodes that observe the data communicate results backward toward the interested nodes. In our
example we choose to protect these pathways with multiple capabilities.

\subsection{Directed Diffusion in Multiple Domains}

The directed diffusion algorithm \cite{intanagonwiwat-2003} is an approach for diffusing data
across a wireless sensor network. The algorithm allows a node to express an \emph{interest} in
data of a certain kind. In our example interests are expressed as temperature thresholds. Any
node that observes a temperature greater than the threshold is requested to report that data
back to the interested node. A certain data rate, expressed as a time interval between
transmissions, is associated with each interest. Initially a node seeking temperature data
floods the network using an interest with a low data rate. As data events find their way back to
the interested node, that node selectively \emph{reinforces} certain immediate neighbors by
retransmitting the interest with a higher associated data rate to just those neighbors.

Each node maintains a cache of active interests. When a node observes or receives a data event
it sends the data to all immediate neighbors that have expressed direct or indirect interest in
it. Since not every neighbor is interested in all data, only a subset of neighbors are involved
in each data transmission.

Each node also maintains a cache of data events that have been recently seen. This cache is
used, in part, to measure the actual rate at which data is received from various neighbors. This
information is made available to the reinforcement algorithm so that an appropriate decision can
be made as to which nodes might be suitable to reinforce.

\subsection{Interfaces}

Interest and data event propagation are handled by separate interfaces, as shown in
Fig.~\ref{figure-ddinterfaces}, each containing a single duty.
\begin{fpfig}[!t]{Directed Diffusion Interfaces}{figure-ddinterfaces}
{
\begin{Verbatim}
interface InterestManagement {
    duty void set_interest(
        uint16_t sender_node_id,
        int      temp_threshold,
        int      interval,
        int      duration);
}

interface DataManagement {
    duty void set_data(
        uint16_t sender_node_id,
        uint16_t originator_node_id,
        int      temp_value)
}
\end{Verbatim}
}
\end{fpfig}

A node expresses interest in temperature data above a certain threshold and at a certain data
rate by posting the \code{set\_interest} duty on its neighboring nodes. Similarly a node passes
data to its interested neighbors by posting the \code{set\_data} duty.

\subsection{Configuration}

The interest and data caches, which we call ``managers,'' are the two central components of our
application. The interest manager provides the \code{Interest\-Management} interface remotely
and uses the same interface on other components. The data manager provides and uses the
\code{Data\-Management} interface in a similar way. Both components serve as their own component
managers, using internal information to specify the destination nodes of each outgoing post
operation.

In our example interest propagation is be controlled by two capabilities. The shared
\code{ext\_interest} capability allows a node from any protection domain to request a low data
rate from nodes in any other domain. The \code{int\_interest} capability is defined internally
and independently by each protection domain, and allows a node in the same domain to request a
high data rate.

The main configuration contains, in part, the following wiring for the interest manager:
\begin{Verbatim}
auth "ext_interest"
InterestManagerC.NeighborSensors ->
    [InterestManagerC].InterestManagement;
auth "int_interest"
InterestManagerC.NeighborSensors ->
    [InterestManagerC].InteresttManagement;
\end{Verbatim}
\vspace{0.3em}

Because the interest manager provides and uses the same interface, it defines
\code{NeighborSensors} as an alias for the \code{Interest\-Management} interface that it uses
remotely. When the interest manager posts the \code{set\_interest} duty, that duty is invoked in
all neighbors \emph{currently} selected by its own, internal component manager. These post
operations are authorized using both interest capabilities; neighbors can be in multiple
protection domains. In this example no attempt is made to track which neighbors are in which
domains. As a result two messages are sent to each neighbor selected by the interest manager,
but this could be improved by using separate component managers for the internal and external
domains.

\subsection{Interest Management}

The interest manager has a partial specification as follows:
\begin{Verbatim}
module InterestManagerC {
    provides interface ComponentManager;
    provides remote interface
        InterestManagement as ExtManagement
            requires "ext_interest";
    provides remote interface
        InterestManagement as IntManagement
            requires "int_interest";
    uses interface InterestManagement
        as NeighborSensors;
}
\end{Verbatim}

The \code{set\_interest} duty is provided for both internal and external post operations. The
implementation is essentially the same. However, the duty used by other protection domains
ignores requests for data rates that are too high.

Because the interest manager is its own component manager, setting up target node addresses
entails updating an internal \code{component\_set} variable as appropriate. In the case when a
new interest is received the interest manager propagates that interest to all neighbors. This is
done inside the interest manager's \code{set\_interest} duty with the following code:

\begin{Verbatim}
remote_set.ids   = &remote_components;
remote_set.count = 1;
remote_components[0].node_id  = 0xFFFF;
remote_components[0].local_id = INTEREST_ID;
post NeighborSensors.set_interest( ... );
\end{Verbatim}
\vspace{0.3em}

The ``well known'' local ID of the interest manager is used to specify which component on the
neighbor nodes is to process the duty. The implementation of the \code{elements} command in the
\code{Component\-Manager} interface merely returns \code{remote\_set} computed above. Before the
posting of \code{set\_interest} returns, \code{remote\_set} is used to prepare the outgoing
packet. After the post is complete \code{remote\_set} and \code{remote\_components} can be
reused without affecting any pending radio transmissions.

In the more complicated case where an interest is being reinforced, the interest manager must
use information in the data cache to compute which neighbors need reinforcing. Although
SpartanRPC allows a component manager to dynamically select neighbor nodes, the component used
as a component manager is statically bound. Thus in this example the interest manager can not
switch its component manager to, for example, the data manager. To work around this, the
interest manager communicates with the data manager using connections not shown here. With the
data manager's help the interest manager computes appropriate neighbors dynamically before
posting \code{set\_interest} on those neighbors.

\subsection{Data Management}

The data manager has a dual structure where the implementation of the \code{set\_data} duty
simply adds the data event to the data cache, and the implementation of a timer fired event
performs the task of propagating data to interested nodes. The data manager manipulates the
timer frequency to match the highest required data rate. However since not all data needs to be
sent to all neighbors at such a high rate, only a dynamically changing subset of neighbors is
selected for each timer event. This is done by adjusting an internal \code{component\_set}
before posting the \code{set\_data} duty.

We further assume that nodes will only want to accept data events from authorized producers. All
legitimate posts of \code{set\_data} must be done using the \code{data} capability. The main
configuration thus also contains dynamic wires such as:
\begin{Verbatim}
auth "data"
DataManagerC.NeighborSensors ->
    [DataManagerC].DataManagement;
\end{Verbatim}
\vspace{0.3em}
The specification of the data manager is, in part:
\begin{Verbatim}
module DataManagerC {
    provides interface ComponentManager;
    provides interface DataManagement
        requires "data";
    uses interface DataManagement
        as NeighborSensors;
}
\end{Verbatim}
Notice that there are no security related artifacts in the body of the data manager's
implementation.
